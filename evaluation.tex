\section{Evaluation}
\label{s:evaluation}

To evaluate our approach, we answer the following research questions:

\begin{itemize}[labelindent=\parindent,leftmargin=*]
\item \textbf{RQ1. Effectiveness}:
How effective is \approach{} at generating reproduction scripts?
\item \textbf{RQ2. Efficiency}:
How efficient is \approach{} in terms of time and monetary cost?
\item \textbf{RQ3. Ablation Study}:
What is the impact of different tools in different configurations?
\item \textbf{RQ4. Fidelity and Accuracy of LLM-Based Tools}:
What is the fidelity of the LLM-based Format tool and the accuracy of the LLM-based Judge tool?
\end{itemize}

\subsection{Experimental Setup}

\subsubsection{Datasets}

For RQ1, RQ2, and RQ3, we use \benchmark{} introduced in Section~\ref{s:benchmark} to evaluate our approach and the baselines.
For RQ4-1, we use 60 ground truth scripts to prepare inputs used to evaluate the fidelity of the Format tool.
For RQ4-2, we use 60 ground truth scripts (30~\fullrepro~scripts, 30~\downrepro~scripts), plus 25~\copyrepro~scripts that we collected during development, to evaluate the accuracy of the Judge tool.

\subsubsection{Baselines}

As baselines, we select mini-swe-agent~\cite{minisweagent} and SWE-agent~\cite{sweagent}.
We choose mini-swe-agent for two reasons:
(1) \approach{} is built on top of mini-swe-agent; thus, comparing \approach{} with mini-swe-agent enables a thorough empirical study of \approach{}'s contributions.
(2) mini-swe-agent achieves competitive performance (74.4\% resolved on SWE-Bench Verified with Claude Opus 4.5, compared to 78.8\% for the top score on the current leaderboard), while remaining simple (138~LOC in the core).
We also include SWE-agent as an established state-of-the-art approach for general software engineering tasks.
For both baselines and~\approach, we experiment with gpt-5.1-2025-11-13, gpt-5-mini-2025-08-07, and gpt-5-nano-2025-08-07 as an LLM model.

\subsubsection{Metrics}

To evaluate effectiveness, we measure the sum of~\fullrepro~scripts and~\downrepro{}~scripts each approach generates.
For the reproduction method we report, we report the result of the manual validation, not the one reported by the Judge tool (Section \ref{s:judge}).
To evaluate efficiency, we report total wall-clock time and LLM token cost per task.
To evaluate the fidelity of the Format tool, we measure what percentage of formatted output matches the expected output with the execution log we manually validated to contain all the expected results given as an input.
To evaluate the accuracy of the Judge tool, we measure what percentage of judgement of reproduction matches with the ground truth reproduction method label we have.
We repeat RQ4 experiments ten times to account for an LLM non-determinism.
\subsection{RQ1: Effectiveness}
\label{s:effectiveness}

\begin{table}[t]
  \caption{Effectiveness of \approach{} on \benchmark{}. GPT-5.1 used for model.}
  \label{t:big_table}
  \begin{tabular}{lccccccc}
\toprule
\multirow{2}{*}{ID} & \multicolumn{2}{c}{Success} & \multicolumn{4}{c}{Failure} & \multirow{2}{*}{\shortstack{Total\\Tasks}} \\
\cmidrule(lr){2-3} \cmidrule(lr){4-7}
 & Full Rep. & Down. Rep. & Copy Res. & Res. Mis. & Run. Err. & Stat. Err. & \\
\midrule
pythonic      & & & & & & & 9 \\
action        & & & & & & & 6 \\
pmsat         & & & & & & & 5 \\
roam          & & & & & & & 5 \\
rust          & & & & & & & 4 \\
unimocg       & & & & & & & 4 \\
crossover     & & & & & & & 3 \\
bloat         & & & & & & & 3 \\
provenfix     & & & & & & & 3 \\
axa           & & & & & & & 3 \\
npetest       & & & & & & & 3 \\
llm           & & & & & & & 2 \\
bazel         & & & & & & & 2 \\
dypybench     & & & & & & & 2 \\
goblinupdater & & & & & & & 2 \\
bcia          & & & & & & & 2 \\
lasapp        & & & & & & & 2 \\
neurojit      & & & & & & & 2 \\
mutation      & & & & & & & 1 \\
ppt4j         & & & & & & & 1 \\
sctype        & & & & & & & 1 \\
inference     & & & & & & & 1 \\
urcrat        & & & & & & & 1 \\
\midrule
Total         & & & & & & & 60 \\
\bottomrule
\end{tabular}
\end{table}

\begin{table}[t]
  \caption{Comparison of \approach{} with baselines.}
  \label{t:Baseline}
  \begin{tabular}{lccccccc}
\toprule
\multirow{2}{*}{ID} & \multicolumn{2}{c}{Success} & \multicolumn{4}{c}{Failure} & \multirow{2}{*}{\shortstack{Cost}} \\
\cmidrule(lr){2-3} \cmidrule(lr){4-7}
 & Full Rep. & Down. Rep. & Copy Res. & Mis. Res. & Run. Err. & Stat. Err. & \\
\midrule

mini-swe-agent & & & & & & & \\
\shortstack{-- w/ gpt-5-nano} & 0 & 0 & 0 &  0 &  0 & 60 & \$0.008 \\
\shortstack{-- w/ gpt-5-mini} & 0 & 0 & 0 & 44 & 14 &  2 & \$0.061 \\
\shortstack{-- w/ gpt-5.1}    &  & & & & & & \\

swe-agent & & & & & & & \\
\shortstack{-- w/ gpt-5-nano} &  0 & 1 & 0 & 34 & 17 & 8 & \$0.061 \\
\shortstack{-- w/ gpt-5-mini} & 34 & 0 & 0 & 23 &  4 & 0 & \$0.136 \\
\shortstack{-- w/ gpt-5.1}   & & & & & & & \\
\midrule

Artisan & & & & & & & \\
\shortstack{-- w/ gpt-5-nano} & & & & & & & \\
\shortstack{-- w/ gpt-5-mini} & & & & & & & \\
\shortstack{-- w/ gpt-5.1}   & & & & & & & \\
\midrule

\bottomrule
\end{tabular}

\end{table}

\subsection{RQ2: Efficiency}
\label{s:efficiency}

\subsection{RQ3: Ablation Study}
\label{s:ablation}

\begin{table}[t]
  \caption{Ablation study of \approach. GPT-5.1 used for all experiments.}
  \label{t:ablation}
  \begin{tabular}{lccccccc}
\toprule
\multirow{2}{*}{ID} & \multicolumn{2}{c}{Success} & \multicolumn{4}{c}{Failure} & \multirow{2}{*}{\shortstack{Total\\Tasks}} \\
\cmidrule(lr){2-3} \cmidrule(lr){4-7}
 & Full Rep. & Down. Rep. & Copy Res. & Res. Mis. & Run. Err. & Stat. Err. & \\
\midrule
mini-swe-agent      & & & & & & & 61 \\
w/o Judge      & & & & & & & 61 \\
w/o Format      & & & & & & & 61 \\
w/o Get      & & & & & & & 61 \\
artisan      & & & & & & & 61 \\
\bottomrule
\end{tabular}
\end{table}

\subsection{RQ4: Fidelity and Accuracy of LLM-Based Tools}
\label{s:llmtools}

\subsubsection{RQ4-1: Fidelity of the Format tool}
\label{s:eval_format}

\subsubsection{RQ4-2: Accuracy of the Judge tool}
\label{s:eval_judge}
