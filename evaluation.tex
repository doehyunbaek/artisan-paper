\section{Evaluation}
\label{s:evaluation}

To evaluate our approach, we answer the following research questions:

\begin{itemize}[labelindent=\parindent,leftmargin=*]
\item \textbf{RQ1. Effectiveness}:
How effective is \approach{} at generating reproduction scripts for tables in software engineering research papers?
\item \textbf{RQ2. Efficiency}:
How efficient is \approach{} in terms of time and monetary cost?
\item \textbf{RQ3. Ablation Study}:
What is the impact of different tools in different configurations?
\item \textbf{RQ4. Effectiveness and Accuracy of LLM-Basesd Tools}:
What is the effectiveness of LLM-based Format tool and accuracy of LLM-based Judge tool?
\end{itemize}

\subsection{Experimental Setup}

\subsubsection{Datasets}

For RQ1, RQ2, RQ3, We use \benchmark{} introduced in Section~\ref{s:benchmark} to evaluate our approach and the baselines.
For RQ4-1, we use 61 ground truth scripts to prepare inputs used to evaluate the effectiveness of Format tool.
For RQ4-2, we use 61 ground truth scripts (31~\fullrepro~scripts, 30~\downrepro~scripts), plus 25~\copyrepro~scripts that we collected during the development to evaluate the accuracy of the Judge tool.

\subsubsection{Baselines}

As baselines, we select mini-swe-agent~\cite{minisweagent} and SWE-agent~\cite{sweagent}.
We choose mini-swe-agent for two reasons:
(1) \approach{} is built on top of mini-swe-agent; thus, comparing \approach{} with mini-swe-agent enables a thorough empirical study of \approach{}'s contributions.
(2) mini-swe-agent achieves competitive performance (70.6\% resolved on SWE-Bench Verified with Claude~4.5 Sonnet, compared to 78.8\% for the top score on the current leaderboard), while remaining simple (131~LOC in the core).
We also include SWE-agent as an established state-of-the-art approach for general software engineering tasks.
For both baseslines and~\approach, we experiment with gpt-5.1-2025-11-13, gpt-5-mini-2025-08-07, and gpt-5-nano-2025-08-07.

\subsubsection{Metrics}

To evaluate effectiveness, we measure the sum of~\fullrepro~scripts and~\downrepro{}~scripts each approaches generate.
For the reproduction method we report, we report the result of the manual validation, not the one reported by the Judge tool (Section \ref{s:judge}).
To evaluate efficiency, we report total wall-clock time and LLM token cost per task.

\subsection{RQ1: Effectiveness}
\label{s:effectiveness}

\begin{table}[t]
  \caption{Effectiveness of \approach{} on \benchmark{}.}
  \label{t:big_table}
  \begin{tabular}{lccccccc}
\toprule
\multirow{2}{*}{ID} & \multicolumn{2}{c}{Success} & \multicolumn{4}{c}{Failure} & \multirow{2}{*}{\shortstack{Total\\Tasks}} \\
\cmidrule(lr){2-3} \cmidrule(lr){4-7}
 & Full Rep. & Down. Rep. & Copy Res. & Res. Mis. & Run. Err. & Stat. Err. & \\
\midrule
pythonic      & & & & & & & 9 \\
action        & & & & & & & 6 \\
pmsat         & & & & & & & 5 \\
roam          & & & & & & & 5 \\
rust          & & & & & & & 4 \\
unimocg       & & & & & & & 4 \\
crossover     & & & & & & & 3 \\
bloat         & & & & & & & 3 \\
provenfix     & & & & & & & 3 \\
axa           & & & & & & & 3 \\
npetest       & & & & & & & 3 \\
llm           & & & & & & & 2 \\
bazel         & & & & & & & 2 \\
dypybench     & & & & & & & 2 \\
goblinupdater & & & & & & & 2 \\
bcia          & & & & & & & 2 \\
lasapp        & & & & & & & 2 \\
neurojit      & & & & & & & 2 \\
mutation      & & & & & & & 1 \\
ppt4j         & & & & & & & 1 \\
sctype        & & & & & & & 1 \\
inference     & & & & & & & 1 \\
urcrat        & & & & & & & 1 \\
\midrule
Total         & & & & & & & 60 \\
\bottomrule
\end{tabular}
\end{table}

\begin{table}[t]
  \caption{Comparison of \approach{} with baselines.}
  \label{t:Baseline}
  \begin{tabular}{lccccccc}
\toprule
\multirow{2}{*}{ID} & \multicolumn{2}{c}{Success} & \multicolumn{4}{c}{Failure} & \multirow{2}{*}{\shortstack{Cost}} \\
\cmidrule(lr){2-3} \cmidrule(lr){4-7}
 & Full Rep. & Down. Rep. & Copy Res. & Mis. Res. & Run. Err. & Stat. Err. & \\
\midrule

mini-swe-agent & & & & & & & \\
\shortstack{-- w/ gpt-5-nano} & 0 & 0 & 0 &  0 &  0 & 60 & \$0.008 \\
\shortstack{-- w/ gpt-5-mini} & 0 & 0 & 0 & 44 & 14 &  2 & \$0.061 \\
\shortstack{-- w/ gpt-5.1}    &  & & & & & & \\

swe-agent & & & & & & & \\
\shortstack{-- w/ gpt-5-nano} &  0 & 1 & 0 & 34 & 17 & 8 & \$0.061 \\
\shortstack{-- w/ gpt-5-mini} & 34 & 0 & 0 & 23 &  4 & 0 & \$0.136 \\
\shortstack{-- w/ gpt-5.1}   & & & & & & & \\
\midrule

Artisan & & & & & & & \\
\shortstack{-- w/ gpt-5-nano} & & & & & & & \\
\shortstack{-- w/ gpt-5-mini} & & & & & & & \\
\shortstack{-- w/ gpt-5.1}   & & & & & & & \\
\midrule

\bottomrule
\end{tabular}

\end{table}

\subsection{RQ2: Efficiency}
\label{s:efficiency}

\subsection{RQ3: Ablation Study}
\label{s:ablation}

\begin{table}[t]
  \caption{Ablation study of \approach.}
  \label{t:ablation}
  \begin{tabular}{lccccccc}
\toprule
\multirow{2}{*}{ID} & \multicolumn{2}{c}{Success} & \multicolumn{4}{c}{Failure} & \multirow{2}{*}{\shortstack{Total\\Tasks}} \\
\cmidrule(lr){2-3} \cmidrule(lr){4-7}
 & Full Rep. & Down. Rep. & Copy Res. & Res. Mis. & Run. Err. & Stat. Err. & \\
\midrule
mini-swe-agent      & & & & & & & 61 \\
w/o Judge      & & & & & & & 61 \\
w/o Format      & & & & & & & 61 \\
w/o Get      & & & & & & & 61 \\
artisan      & & & & & & & 61 \\
\bottomrule
\end{tabular}
\end{table}
