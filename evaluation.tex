\section{Evaluation}
\label{s:evaluation}

To evaluate our approach, we answer the following research questions:

\begin{itemize}[labelindent=\parindent,leftmargin=*]
\item \textbf{RQ1. Effectiveness}:
How effective is \approach{} at generating reproduction scripts for the tables of software engineering research papers?
\item \textbf{RQ2. Efficiency}:
How efficient is \approach in terms of time and monetary cost?
\item \textbf{RQ3. Ablation Study}:
What is the impact of different tools in different configuartions?
\end{itemize}

\subsection{Experimental Setup}

\subsubsection{Datasets}

We use SEA-Bench introduced in Section~\ref{s:benchmark} to evaluate our approach and the baselines.

\subsubsection{Implementation and Model}

We build~\approach~on top of mini-swe-agent\cite{minisweagent,sweagent}.
To implement the tools, we use Python 3.12.10.
We use GPT-5-mini-2025-08-07 as our model.

\subsubsection{Baslines}

As our baselines, we select mini-swe-agent\cite{minisweagent} and SWE-agent\cite{sweagent}.
We choose mini-swe-agent as one of our baselines for the two reasons:
(1) \approach~is built on top of mini-swe-agent. Thus, comparing \approach with mini-swe-agent allows thorough empirical study of the contributions of~\approach.
(2) mini-swe-agent has competitive performance (70.6\% resolved at SWE-Bench Verified with Claude 4.5 Sonnet, compared to 78.8\% which is the top score at the current leaderboard), despite being very simple (131 LOC in terms of core).
We choose SWE-agent as it is the established state-of-the-art approach for the general software engineering tasks.
We use the same GPT-5-mini-2025-08-07 for models of mini-swe-agent and SWE-agent.

\subsubsection{System Prompt Used}

To implement Artisan, we use system prompt developed for the purpose of artifact evalution~\footnote{https://github.com/doehyunbaek/artisan//blob/9b13d07f/src/artisan/mini.py\#L22-\#L71}.
This is combined with the user prompt described before at~\ref{s:user-prompt} to constitue an input to the agent.
For other baselines, we use default system prompt for SWE-Agent\footnote{https://github.com/SWE-agent/SWE-agent/blob/8089c8ba/config/default.yaml\#L7} for SWE-Agent, and default system prompt for mini-swe-agent\footnote{https://github.com/SWE-agent/mini-swe-agent/blob/eeaf79b4/src/minisweagent/config/default.yaml\#L2-\#L17} for mini-swe-agent.

\subsubsection{Metrics}

To evaluate the effectiveness, we use Best-of-N metric with N set to 10.
That is, we allow each agent to attempt to reproduce the table N times sequentially, terminating early if one of the attempt leads to successful reproduction.
To evaluate efficiency, we report total time and LLM token cost spent per task.

\begin{table}[t]
  \caption{Effectiveness of Artisan on the \benchmark}
  \label{t:big_table}
  \begin{tabular}{lccccccc}
\toprule
\multirow{2}{*}{ID} & \multicolumn{2}{c}{Success} & \multicolumn{4}{c}{Failure} & \multirow{2}{*}{\shortstack{Total\\Tasks}} \\
\cmidrule(lr){2-3} \cmidrule(lr){4-7}
 & Full Rep. & Down. Rep. & Copy Res. & Res. Mis. & Run. Err. & Stat. Err. & \\
\midrule
pythonic      & & & & & & & 9 \\
action        & & & & & & & 6 \\
pmsat         & & & & & & & 5 \\
roam          & & & & & & & 5 \\
rust          & & & & & & & 4 \\
unimocg       & & & & & & & 4 \\
crossover     & & & & & & & 3 \\
bloat         & & & & & & & 3 \\
provenfix     & & & & & & & 3 \\
axa           & & & & & & & 3 \\
npetest       & & & & & & & 3 \\
llm           & & & & & & & 2 \\
bazel         & & & & & & & 2 \\
dypybench     & & & & & & & 2 \\
goblinupdater & & & & & & & 2 \\
bcia          & & & & & & & 2 \\
lasapp        & & & & & & & 2 \\
neurojit      & & & & & & & 2 \\
mutation      & & & & & & & 1 \\
ppt4j         & & & & & & & 1 \\
sctype        & & & & & & & 1 \\
inference     & & & & & & & 1 \\
urcrat        & & & & & & & 1 \\
\midrule
Total         & & & & & & & 60 \\
\bottomrule
\end{tabular}
\end{table}