\section{Evaluation}
\label{s:evaluation}

To evaluate our approach, we answer the following research questions:

\begin{itemize}[labelindent=\parindent,leftmargin=*]
\item \textbf{RQ1. Effectiveness}:
How effective is \approach{} at generating successful reproduction scripts?
\item \textbf{RQ2. Efficiency}:
How efficient is \approach{} in terms of monetary cost and execution time?
\item \textbf{RQ3. Ablation Study}:
What is the impact of different tools in different configurations?
\item \textbf{RQ4. Performance and Accuracy of LLM-Based Tools}:
What is the performane of the LLM-based Format tool and the accuracy of the LLM-based Judge tool?
\end{itemize}

\subsection{Experimental Setup}

\subsubsection{Datasets}

For RQ1, RQ2, and RQ3, we use \benchmark{} introduced in Section~\ref{s:benchmark} to evaluate our approach and the baselines.
For RQ4-1, we use 60 ground truth scripts to prepare execution logs used to evaluate the performane of the Format tool.
We manually validate all execution logs to contain all the expected results.
For RQ4-2, we use 60 ground truth scripts (30~\fullrepro~scripts, 30~\downrepro~scripts), plus~\fastpathsize~\copyrepro~scripts that we collected during development, to evaluate the accuracy of the Judge tool.

\subsubsection{Baselines}

\begin{figure}[t]
\begin{minted}{diff}
> - `artisan get`: Download and extract the artifact from the official URL.
> - `artisan format --expected expected.md --repro repro.txt`: Render final tables.
> - `artisan submit`: Submit your reproduction script for evaluation.
>    - You MUST use`artisan get {{ artifact_url }}`.
> artisan get {{ artifact_url }}
> artisan format --expected expected.md --repro repro.txt
> chmod +x /workspace/repro_{{ artifact_name }}_table_{{ table_index }}.sh
> artisan submit /workspace/repro_{{ artifact_name }}_table_{{ table_index }}.sh
\end{minted}
\caption{Diff of user prompts between mini-swe-agent and \approach.}
\Description{Diff of user prompts between mini-swe-agent and \approach.}
\label{s:line_diffs}
\end{figure}

As baselines, we select SWE-agent~\cite{sweagent} and mini-swe-agent~\cite{minisweagent}.
We choose SWE-agent as an established state-of-the-art approach for general software engineering tasks.
We choose mini-swe-agent as~\approach{} is built on top of mini-swe-agent.
Comparing \approach{} with mini-swe-agent enables a thorough empirical study of \approach{}'s contributions.
% (2) mini-swe-agent achieves competitive performance (74.4\% resolved on SWE-Bench Verified with Claude Opus 4.5, compared to 78.8\% for the top score on the current leaderboard), while remaining simple (138~LOC in the core).
For both baselines and~\approach, we experiment with gpt-5.1-2025-11-13, gpt-5-mini-2025-08-07, and gpt-5-nano-2025-08-07 as an LLM model.
For both baselines and~\approach, we use the step limit of 30 and cost limit of one dollar per task.

To adopt mini-swe-agent to the task of~\benchmark, we give the same user prompt as in~\approach, minus the three tools that we present in Section~\ref{s:approach} and remove the validation on exit mechanism introduced in~\ref{s:implementation}.
We present the diffs of the user prompt between mini-swe-agent and~\approach in Figure~\ref{s:line_diffs}.
To adopt SWE-agent to the task of~\benchmark, we give the same user prompt as in mini-swe-agent but adds one last instruction to the workflow (``6. Copy the submission script: \verb|cp repro_{{ artifact_name }}_table_{{ table_index }}.sh /root/model.patch|'').
This is to adopt SWE-agent's hard-coded APR task to our task.


\subsubsection{Metrics}

To evaluate effectiveness, we classify the submitted script by the baselines and~\approach~into six categories, following the classification introduced in Section~\ref{s:judge}.
For three categories where results of the submitted script fail to match the expected result, we automatically classify them into the three categories based on the execution outputs (Section~\ref{s:judge_execution}).
For three categories where results of the submitted script match the expected result, we manually classify the type of the reproduction method (Section~\ref{s:judge_llm}).
For the single metric of effectiveness, we report the sum of~\fullrepro~scripts and~\downrepro{}~scripts each approach generates, which we consider to be successful reproduction.

To evaluate efficiency, we report monetary cost per task and wall-clock time per task.
Monetary cost includes the cost of the Format tool (Section~\ref{s:format}) and Judge tool (Section~\ref{s:judge}).
Note also that when we measure time, it is an end-to-end time including the running time of experiments themselves.
Thus, if a successful agent run involves running experiments that take few hours, that would lead to increase in average time.

To evaluate the performane of the Format tool, we measure what percentage of formatted output matches the expected output.
To evaluate the accuracy of the Judge tool, we measure what percentage of classification of reproduction method matches with the ground truth.
% We repeat RQ4 experiments ten times to account for an LLM non-determinism.

\subsection{RQ1: Effectiveness}
\label{s:effectiveness}
% \begin{table}[t]
%   \caption{Effectiveness of \approach{} on \benchmark{}. GPT-5.1 used for model.}
%   \label{t:big_table}
%   \begin{tabular}{lccccccc}
\toprule
\multirow{2}{*}{ID} & \multicolumn{2}{c}{Success} & \multicolumn{4}{c}{Failure} & \multirow{2}{*}{\shortstack{Total\\Tasks}} \\
\cmidrule(lr){2-3} \cmidrule(lr){4-7}
 & Full Rep. & Down. Rep. & Copy Res. & Res. Mis. & Run. Err. & Stat. Err. & \\
\midrule
pythonic      & & & & & & & 9 \\
action        & & & & & & & 6 \\
pmsat         & & & & & & & 5 \\
roam          & & & & & & & 5 \\
rust          & & & & & & & 4 \\
unimocg       & & & & & & & 4 \\
crossover     & & & & & & & 3 \\
bloat         & & & & & & & 3 \\
provenfix     & & & & & & & 3 \\
axa           & & & & & & & 3 \\
npetest       & & & & & & & 3 \\
llm           & & & & & & & 2 \\
bazel         & & & & & & & 2 \\
dypybench     & & & & & & & 2 \\
goblinupdater & & & & & & & 2 \\
bcia          & & & & & & & 2 \\
lasapp        & & & & & & & 2 \\
neurojit      & & & & & & & 2 \\
mutation      & & & & & & & 1 \\
ppt4j         & & & & & & & 1 \\
sctype        & & & & & & & 1 \\
inference     & & & & & & & 1 \\
urcrat        & & & & & & & 1 \\
\midrule
Total         & & & & & & & 60 \\
\bottomrule
\end{tabular}
% \end{table}
\begin{table}[t]
  \caption{Comparison of \approach{} with baselines.}
  \label{t:baseline}
  \begin{tabular}{lccccccc}
\toprule
\multirow{2}{*}{ID} & \multicolumn{2}{c}{Success} & \multicolumn{4}{c}{Failure} & \multirow{2}{*}{\shortstack{Cost}} \\
\cmidrule(lr){2-3} \cmidrule(lr){4-7}
 & Full Rep. & Down. Rep. & Copy Res. & Mis. Res. & Run. Err. & Stat. Err. & \\
\midrule

mini-swe-agent & & & & & & & \\
\shortstack{-- w/ gpt-5-nano} & 0 & 0 & 0 &  0 &  0 & 60 & \$0.008 \\
\shortstack{-- w/ gpt-5-mini} & 0 & 0 & 0 & 44 & 14 &  2 & \$0.061 \\
\shortstack{-- w/ gpt-5.1}    &  & & & & & & \\

swe-agent & & & & & & & \\
\shortstack{-- w/ gpt-5-nano} &  0 & 1 & 0 & 34 & 17 & 8 & \$0.061 \\
\shortstack{-- w/ gpt-5-mini} & 34 & 0 & 0 & 23 &  4 & 0 & \$0.136 \\
\shortstack{-- w/ gpt-5.1}   & & & & & & & \\
\midrule

Artisan & & & & & & & \\
\shortstack{-- w/ gpt-5-nano} & & & & & & & \\
\shortstack{-- w/ gpt-5-mini} & & & & & & & \\
\shortstack{-- w/ gpt-5.1}   & & & & & & & \\
\midrule

\bottomrule
\end{tabular}

\end{table}

The left-hand side of Table~\ref{t:baseline} shows the results for effectiveness.
\approach~with gpt-5.1 gives the best result, successfully reproducing the expected table in~\goodscripttsize~ / 60 tasks (20 full reproduction and 21 last-mile reproduction).
The second-best result is from~\approach~with gpt-5-mini with successful reproduction of 26 / 60 (12 full reproduction and 14 last-mile reproduction), even outperforming baselines with gpt-5.1.
\approach~with gpt-5-nano effectively decreases the rate of static error to 8 / 60 (compared to 49 / 60 for SWE-agent and 60 / 60 for mini-swe-agent) but does not lead to substantial increase in successful reproduction (only 1 / 60 success).
Manual analysis of trajectories of~\approach~with gpt-5-nano reveals that gpt-5-nano significantly lacks the capacity to follow the given instruction, leading to more failures in the steps and quickly reaching the step limit (59 / 60).
Still, the presence of Judge tool lets it avoid static errors at least.

We manually analyze 19 cases where~\approach~with gpt-5.1 fails to reproduce the expected table.
Our manual analysis reveals that for 16 / 19 tasks which~\approach~fails to reproduce expected tables,~\approach~exceeds the step limit of 30 and exits.
This is due to our mechanism of running validation on exit attempt (Section~\ref{s:implementation}) which forces the agent to keep trying until obtaining successful result.
For 3 / 19 tasks, the submission passed the automated validation but still fails to reproduce the expected results.
2 tasks failed due to the non-determinism of LLM-based Format tool and 2 tasks failed to fail due to the non-determinism of LLM-based Judge tool.
For 7 / 16 tasks, Get tool fails to download the artifact due to network error.
As we target the live artiafct repositories as our target, some of the instabilities are inavoidable.
\footnote{During the development, we even encountered a Cloudflare outrage that prevented us from evaluating our approach \cite{cloudflare}.}
Although disabled for fair evaluation, the caching mechanism of the Get tool (Section~\ref{s:get}) would help mitigate the network errors for the in-the-wild usage.
For 3 / 16 tasks, LLM struggles with instruction following and quickly wastes the remaining steps by running \emph{echo} commands.
For 2 / 16 tasks, docker runtime exits for unknown reason.
For 4 / 16 reamining tasks,~\approach~fails to find successful~\goodscript~given the budget.

\todo{Redo the SWE-agent experiments}
We also manually analyze 45 cases where mini-swe-agent with gpt-5.1 fails to reproduce the expected table, which is discussed in Section~\ref{s:ablation}.
Here, we report the manual analysis of 58 cases where SWE-agent with gpt-5.1 fails to reproduce the expected table.
For all 42 static errors that SWE-agent encounter, it is due to SWE-agent failing to create the submission script.
For 24 / 42 of such cases, SWE-agent errors while making request to the agent runtime.
For 18 / 42, SWE-agent exceeds the step limit before creating the script for submission.
For 5 tasks that SWE-agent generated script encounter runtime-error, three tasks are terminated due to step limit and two tasks are submitted within step limits but still fail.
For all 11 tasks that SWE-agent generated script does not match with the expected table, SWE-agent submits a script that runs without any errors but the results differ.

\subsection{RQ2: Efficiency}
\label{s:efficiency}

The right-hand side of Table~\ref{t:baseline} shows the results for efficiency.
In general, the trend is that to be more effective, you have to pay more in cost and time.
\approach~is consistently the most expensive and time-consuming approach given the same model.

Comparing~\approach~with less powerful LLMs against baselines with stronger LLMs show an interesting trend.
\approach~with gpt-5-mini is more efficient in terms of cost, costing \$0.137 per instance, compared to mini-swe-agent with gpt-5.1 (\$0.285 per instance), while being more effective.
\approach~with gpt-5-nano is more efficient in terms of cost, costing \$0.061 per instance, compared to SWE-agent with gpt-5-mini (\$0.086 per instance), while being slightly more effective.
This shows that the use of~\approach~leads to parento frontier not only in terms of effectiveness, but also in efficient but moderately effective part.

One potential room for improvement for efficiency of~\approach~is cost of Format tool.
While LLM-based Judge tool incurs minimal cost (less than \$0.001 per instance), Format tool incurs substantial cost, accounting for 52\% of total cost for~\approach~~with gpt-5-nano , 37\% for~\approach~with gpt-5-mini, 3\% for~\approach~with gpt-5.1.
Although we chose gpt-5-mini as a LLM of choice to implement Format tool informed by our experiment in Section~\ref{s:eval_format}, this choice can be revisited depending on the use cases.

\subsection{RQ3: Ablation Study}
\label{s:ablation}

\begin{table}[t]
  \caption{Ablation study of \approach. GPT-5.1 used for all experiments.}
  \label{t:ablation}
  \begin{tabular}{lccccccc}
\toprule
\multirow{2}{*}{ID} & \multicolumn{2}{c}{Success} & \multicolumn{4}{c}{Failure} & \multirow{2}{*}{\shortstack{Total\\Tasks}} \\
\cmidrule(lr){2-3} \cmidrule(lr){4-7}
 & Full Rep. & Down. Rep. & Copy Res. & Res. Mis. & Run. Err. & Stat. Err. & \\
\midrule
mini-swe-agent      & & & & & & & 61 \\
w/o Judge      & & & & & & & 61 \\
w/o Format      & & & & & & & 61 \\
w/o Get      & & & & & & & 61 \\
artisan      & & & & & & & 61 \\
\bottomrule
\end{tabular}
\end{table}

Table~\ref{t:ablation}~gives the result of our ablation study.
As~\approach's difference with mini-swe-agent baseline is difference in user prompt described in Figure~\ref{s:line_diffs} with the validation on exit mechanism introduced in~\ref{s:implementation}, this lets us thorough study of the contributions of parts of our approach.
For the ablation study, we prepare three additional variants of~\approach, excluding one tool at a time.

To start, we present a result of manual analysis of failure cases of mini-swe-agent \todo.

\approach~without Judge tool is the worst performing variant, \todo.

\approach~without Format tool is the next variant, \todo.

\approach~without Get tool is the final variant, \todo.

To conclude, all tools in~\approach~contribute to the overall effectiveness of~\approach, which is shown by the difference of reproduction scripts produced and analysis.

\subsection{RQ4: Performance and Accuracy of LLM-Based Tools}
\label{s:llmtools}

\begin{table}[ht]
    \centering
    \begin{minipage}{0.48\textwidth}
        \centering
        \caption{Performance of LLM-based Format Tool}
        \label{t:performance}
        \begin{tabular}{lccc}
            \toprule
            \textbf{Model} & \textbf{Success Rate} & \textbf{Cost} & \textbf{Time} \\
            \midrule
            gpt-5-nano & 15.4\% & \$0.0050 & 86.83s \\
            gpt-5-mini & 67.3\% & \$0.0062 & 44.17s \\
            gpt-5.1 & - & - & - \\
            \bottomrule
\end{tabular}
    \end{minipage}
    \hfill
    \begin{minipage}{0.48\textwidth}
        \centering
        \caption{Accuracy of LLM-based Judge Tool}
        \label{t:confusion}
        \begin{tabular}{l|ccc}
            \toprule
            \textbf{Actual} $\backslash$ \textbf{Pred} & \textbf{Full} & \textbf{Last} & \textbf{Copy} \\
            \midrule
            \textbf{Full} & 26 & 2 & 2 \\
            \textbf{Last-Mile} & 9 & 20 & 1 \\
            \textbf{Copy} & 0 & 6 & 24 \\
            \bottomrule
\end{tabular}
    \end{minipage}
\end{table}

\subsubsection{RQ4-1: Performance of the Format tool}
\label{s:eval_format}

Table~\ref{t:performance} gives results for the performance of the Format tool depending on the LLM model used.
\todo{analyze the gpt-5-mini failures}

\subsubsection{RQ4-2: Accuracy of the Judge tool}
\label{s:eval_judge}

Table~\ref{t:confusion} gives results for the accuracy of the LLM-based Judge tool.
Overall, LLM-based Judge tool has an accuracy of 78\% (70 / 90), which we find to be accurate.
Other things to note is the ratio of successful reproductions (full, last-mile) being classifed as~\copyrepro, which is just 5\% (3 / 60).
We find this be low enough to be hinder the progress not so much.
Furthermore, even with this kind of misclassification,~\approach~can still make progress by trying to generate~\goodscript~which is more complete.
The ratio of~\copyrepro~being classified as the successful reproductions is 20\% (6 / 30), which is quite high.
Still, as we manually validate all final~\goodscript~for their reproduction method, we find this to be acceptable in practice.
