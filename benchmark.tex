\section{Benchmark}
\label{s:benchmark}

In this section, we present the methodology used to construct \benchmark.
We first state the requirements we target (Section~\ref{s:requirements}), then describe paper selection (Section~\ref{s:paper_select}) and task selection (Section~\ref{s:task_select}).

\subsection{Requirements}
\label{s:requirements}

We impose two core requirements on \benchmark.
(1) Tasks should closely reflect real-world artifact evaluation so that performance on the benchmark generalizes to practical use.
This addresses limitations of prior benchmarks~\cite{super,corebench,reprobench}, which we find do not fully mirror artifact evaluation as practiced in computer science research.
(2) Manual validation is required.
Although SWE-Bench~\cite{jimenez2024swebench} is influential for evaluating LLMs on real software engineering issues, subsequent efforts such as SWE-Bench Verified~\cite{chowdhury2024swebenchverified} and Wang et al.~\cite{DBLP:journals/corr/abs-2503-15223} highlight the importance of human verification to avoid overestimating or underestimating agent capabilities.
These requirements introduce trade-offs.
For example, the task suite must be broad enough to cover diverse use cases yet small enough to permit manual validation.
Further, a very large task set size increases time and monetary costs, which can be prohibitive for resource-constrained academic groups~\cite{DBLP:conf/iclr/ChanCJASMSLMPMW25}.

\subsection{Paper Selection}
\label{s:paper_select}

\begin{table}[t]
  \caption{Paper selection criteria.}
  \label{t:paper_select}
  \centering
  \begin{tabular}{@{}l r@{}}
    \toprule
    Criterion & Count \\
    \midrule
    Flagship SE conference papers (2024) with \emph{Available} and \emph{Reusable} badges & 114 \\
    Packaged using Docker & 72 \\
    No non-public API use & 63 \\
    No GPU use & 44 \\
    Less than eight hours per task & 32 \\
    Manual reproduction successful & \papersetsize \\
    \bottomrule
  \end{tabular}
\end{table}

\begin{table}[t]
  \centering
  \caption{Distribution of Software Engineering Techniques and Programming Languages.}
  \label{t:paper_distribution}
  \begin{tabular}{lclc}
    \toprule
    \textbf{Technique} & \textbf{Papers} & \textbf{Language} & \textbf{Papers} \\
    \midrule
    Empirical Study             & 5 & Python & 13 \\
    Static Analysis             & 5 & Java   & 4  \\
    Security                    & 2 & Rust   & 2  \\
    Dynamic Analysis            & 2 & Scala  & 2  \\
    Software Maintenance        & 2 & OCaml  & 1  \\
    Others                      & 7 & Bash   & 1  \\
    \midrule
    \textbf{Total}              & \textbf{23} & \textbf{Total} & \textbf{23} \\
    \bottomrule
  \end{tabular}
\end{table}

Guided by these requirements, we curate a set of \papersetsize~papers following the process in Table~\ref{t:paper_select}.
(1) We gather 114 papers from four flagship software engineering venues (ICSE, FSE, ASE, ISSTA) in 2024 with ACM \emph{Artifacts Available} and \emph{Artifacts Evaluated—Reusable} badges~\cite{acm-artifact-badging-v1_1}.
We then manually locate links to the accompanying artifacts.
For FSE and ISSTA, the ACM Digital Library lists artifact links; for other venues, we use a web search (e.g., Google) with the paper title.
(2) We exclude 42 papers whose artifacts are not packaged using Docker.
This reflects the fact that Docker and related virtualization technologies are now standard in artifact evaluation~\cite{icse2024-ae,fse2024-ae,ase2024-ae}.
To implement this, we case-insensitively search Markdown and PDF files for the term ``docker''.
% \footnote{\url{https://github.com/doehyunbaek/artisan/blob/bc7e6cd/evaluation/select_papers.py\#L234}}
(3) We exclude nine papers that rely on non-public APIs.
To implement this, we search Python files and Jupyter notebooks for ``openai'' or ``anthropic''.
We also search Markdown files for ``etherscan''.
This excludes seven papers using OpenAI APIs and two papers using Etherscan APIs.
% .\footnote{\url{https://github.com/doehyunbaek/artisan/blob/bc7e6cd/evaluation/select_papers.py\#L276}}
(4) We exclude 19 papers that require GPUs or specialized hardware.
This is to maintain accessibility across hardware setups.
We search for common GPU keywords (e.g., \texttt{cuda}, \texttt{cudnn}, \texttt{tensorflow}, \texttt{nvidia-smi}, \texttt{gpus}) while ignoring non-code extensions (e.g., \texttt{.txt}, \texttt{.csv}, \texttt{.json}, \texttt{.log}).
% \footnote{\url{https://github.com/doehyunbaek/artisan/blob/bc7e6cd/evaluation/select_papers.py\#L312}}
This aligns with previous work~\cite{super} that excludes papers that require GPU use due to cost.
(5) We exclude twelve papers that require more than eight hours to reproduce.
We search the paper and artifact for the keyword ``hour'' and exclude papers that explicitly mention experiment time beyond eight hours.
Out of twelve papers excluded, four papers take more than 12 hours, seven papers take more than 24 hours, and one paper takes more than 125 hours.
This time requirement is much higher than previous work (10 minutes for SUPER~\cite{super}), which might make the evaluation longer.
We motivate this decision by the need to cover diverse software engineering techniques, where many take hours of compute to output meaningful outcomes, even partial ones.
(6) Finally, we exclude nine papers that we could not manually reproduce within an eight-hour budget.
Out of the nine papers excluded, two are excluded because most of the evaluation dataset is missing, two because raw data is missing, two because the script to generate results from raw data is missing, two because the outputs are non-deterministic, and one because the Docker image mentioned in the artifact is no longer accessible.

Table~\ref{t:paper_distribution} shows the distribution of software engineering techniques and programming languages covered by our paper selection.
For software engineering techniques, \benchmark~covers diverse techniques not covered by existing benchmarks.
Papers in the ``Others'' category include one each for Fuzzing, Mutation Testing, Model-based Testing, Mobile Applications, Program Repair, Test Automation, and AI for Software Engineering.
One thing to note is a possible under-representation of AI4SE and SE4AI techniques.
This is mainly due to our exclusion of papers using non-public APIs and GPUs.
Still, there is one paper~\cite{DBLP:conf/kbse/LeeJL24} that makes use of classical machine learning techniques (Random Forest and XGBoost) and does not use either non-public APIs or GPUs.
For programming languages, Python is the most prevalent, comprising 57\% of the papers.
However, there is a sizable representation of other languages, notably Java, Rust, OCaml, Scala, C, and even Bash, which were not covered in previous benchmarks~\cite{super, corebench, reprobench}.

\subsection{Task Selection}
\label{s:task_select}

\begin{table}[t]
  \caption{Task selection criteria.}
  \label{t:task_select}
  \centering
  \begin{tabular}{@{}l r@{}}
    \toprule
    Criterion & Count \\
    \midrule
    Total number of tables & 89 \\
    Exclude non-results & 68 \\
    Exclude missing code & 62 \\
    Exclude non-deterministic only & \tablesetsize \\
    \bottomrule
  \end{tabular}
\end{table}

\begin{table}[t]
  \centering
  \caption{Distribution of inconsistent tables and partial tables.}
  \label{t:task_distribution}
  \begin{tabular}{lclc}
    \toprule
    \textbf{Consistency} & \textbf{Tasks} & \textbf{Original Tables} & \textbf{Tasks} \\
    \midrule
    Consistent Table    & 38 & Original Table  & 41 \\
    Inconsistent Table  & 23 & Partial Table   & 20 \\
    \midrule
    \textbf{Total}              & \textbf{\tablesetsize} & \textbf{Total} & \textbf{\tablesetsize} \\
    \bottomrule
  \end{tabular}
\end{table}

From the \papersetsize~papers selected, we curate \tablesetsize~tasks that are given to LLM agents as a single task instance.
We start with 89 tables contained in the~\papersetsize~papers.
(1) We exclude 21 non-result tables.
Out of the 21 non-result tables excluded, 14 tables contain descriptions of evaluation setups, and 7 tables contain descriptions of the approach.
(2) We exclude six tables that were missing code to reproduce the results.
For all six tables, code reproducing the other parts of the experiment is present, but the code for some parts is missing.
(3) We exclude one table that only contains non-deterministic results.
For PPT4J~\cite{DBLP:conf/icse/Pan00Z0024}, Table 3, the table reports only the time consumption of the approaches.
As~\approach~does not handle non-deterministic results such as time consumption (Section~\ref{s:limitations}), we exclude this table.
For the remaining~\tablesetsize~tasks, we reproduce the tables and obtain ground truth scripts that reproduce the table.

Table~\ref{t:task_distribution} shows whether the tables used for the tasks are consistent or partial with respect to the tables in the paper.
23 tables that we use for the benchmark are inconsistent with the table in the paper.
For 19 tasks, this is due to~\newbug~that we found, which we discuss in detail in Section~\ref{s:inconsistencies}.
For three, it is because the paper authors claim the results of the artifact could be slightly different from those in the paper~\cite{DBLP:journals/pacmse/Song0LCR24}, and for one, it is because the implementation used for the paper~\cite{DBLP:conf/kbse/Bock0C24} contains a bug, which is fixed now.
20 tables that we use for the benchmark contain partial entries compared to the table in the paper.
For 15, we could only partially reproduce the results of the original table with our ground truth scripts.
For 5, we exclude the parts where the output is non-deterministic.


% \textbf{Ground Truth Kind} & \textbf{Tasks} \\
% & Full Reproduction       & 31 \\
% & Downstream Reproduction & 30 \\

% \begin{figure}[t]
%   \centering
%   \begin{minted}{bash}
% #!/usr/bin/bash
% # Section 1: Write the expected table to /workspace/expected.md
% cat > /workspace/expected.md << 'EOTABLE'
% **Table 2. Ping-pong server: dominant reachable states, glitches, and frequency statistics.**

% | (n) | (n_{reach}) | # Glitches | Mean (\delta_g) fr. | Max (\delta_g) fr. | Min (\delta) fr. |
% | --: | ----------: | ---------: | ------------------: | -----------------: | ---------------: |
% |   3 |           ? |         ?? |                ?.?? |                 ?? |               ?? |
% |   4 |           ? |          ? |                ?.?? |                  ? |               ?? |

% EOTABLE
% # Section 2: Download and extract the artifact
% artisan get https://zenodo.org/records/10423670
% # Section 3: Run the commands to reproduce the results
% # Use docker-compose to run the pmsat container, ensure numpy<2 for compatibility, run mining and parse to produce /workspace/repro.txt
% docker-compose -f pmsat-inference-and-publication-artifacts/pmsat-inference/docker-compose.yml run --rm pmsat /bin/bash -c "cd /pmsat-inference && python -m pip install 'numpy<2' && python run_pmsat_on_traces.py examples-results/ping_pong_example/info.json -nmax 7 && MPLBACKEND=Agg python parse_single_run_results.py TRACE-results/92e710ef352c4739cd7569794272588a" > /workspace/repro.txt
% # Section 4: Format the result into the expected table with artisan format and surround with the required <artisan_submit> block
% echo '<artisan_submit>'
% artisan format --expected /workspace/expected.md --repro /workspace/repro.txt
% echo '</artisan_submit>'
%   \end{minted}
%   \caption{Example~\goodscript~generated by~\approach.}
%   \Description{Example~\goodscript~generated by~\approach.}
%   \label{f:good_script}
% \end{figure}
