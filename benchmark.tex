\section{Benchmark}
\label{s:benchmark}

\subsection{Overview}

The notable difference between our work and previous work is that we are script-based.
Why is it important?
It is immediately useful. It acts as a standalone script that is push-button runnable which reproduces the particular table.
It is supported by a concrete execution.
What’s a good word to capture this? Realistic? Execution-based? Execution-proof? Evidence-accompanying?
Comparison with existing work in that they formulate reproduction as a prediction problem but we as a code generation problem?
It (loosely) implies that this can be used to correct mistakes made in the paper.
Previous works have not pointed out any errors in the paper.
We don’t need separate ground truth. Numbers in the paper is the ground truth.
Less prone to data combination?
Or it doesn’t matter if it’s contaminated because it’s useful?
What’s the argument for this?

\subsection{Paper Selection}
- Top SE (ICSE, FSE, ASE, ISSTA) with Available and Reusable badges 2024: 114
- Find artifact url
- Packaged using Docker: 77
- Exclude Commercial LLM API use Github token (non-public): 67
- Exclude GPU use: 46
- ess than 8 hours per task (when they mention it): 40

\subsection{Task Selection}
We want our benchmark to be
- No cherry-picking
- Extensive (100~300) but not too expensive and long-running to run.
- Manual validation is desired but should not take too much time.

There are few different options:
Run full task set without modification (162)
Manually validate all task set (162) to filter out unreproducible and keep the partial table
Random sample to smaller size (25 / 50)

I argue for the second option (manually validate all task sets).
It is not too much work (1 hour per task) to check whether it is sensible to reproduce the table by running artisan.
We have about 127 tables to manually validate and we can sensibly validate them by the end of the ISSTA submission
Makes our benchmark higher quality. Less underestimation of agent capability
If we manually validate all without bias, no cherry-picking.
Is using artisan as a time-saving measurement already a bias?

Randomly sample and manually validate and time budget with time budget.
