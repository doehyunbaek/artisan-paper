\documentclass[acmsmall,screen,review,anonymous]{acmart}

% Minimal, venue-agnostic packages (no project-specific macros)
\usepackage[capitalise,nameinlink]{cleveref}
\usepackage[many]{tcolorbox}
\usepackage{listings}
\usepackage{enumitem}
\usepackage{pifont}
\usepackage{subcaption}
\usepackage{algorithm}
\usepackage{algpseudocode}
\usepackage{booktabs}
\usepackage{tabularx}
\usepackage{multirow}
\usepackage{makecell}

\begin{CCSXML}
<ccs2012>
   <concept>
       <concept_id>10011007.10011006.10011073</concept_id>
       <concept_desc>Software and its engineering~Software maintenance tools</concept_desc>
       <concept_significance>500</concept_significance>
       </concept>
 </ccs2012>
\end{CCSXML}
\ccsdesc[500]{Software and its engineering~Software maintenance tools}
\keywords{Artifact Evaluation, Benchmarks, LLM agents}

\begin{document}

\newcommand\benchmark{SEAE-Bench}

\newcommand\artisanpartial{\textcolor{red}{TODO}}
\newcommand\artisanpartialoutperform{\textcolor{red}{TODO}}
\newcommand\inconsistencies{\textcolor{red}{TODO}}

\title{Artisan: Automated Artifact Evaluation with Agents}
\begin{abstract}
    Artifact evaluation has beenproven to be an invaluable resource in ensuring the reproduction of software artifacts accompanying research.
    But, artifact evaluation practiced in its current form is notoriously labor-intensive.
    To mitigate this, there has been a recent effort to benchmark the performance of LLM agents on reproducing research results.
    We find several limitations with the existing benchmarks:
    (1) Existing benchmarks largely formulate the problem as an output prediction problem, which undermines its credibility.
    (2) Due to such problem formulation, they have limited usefulness: Contrary to human artifact evaluation which adds confidence that the result of papers are reproducible, exercise of benchmarks do not confer similar confidence.
    (3) They are limited in terms of domain of computer science (ML, NLP) and programming languages (Python, R, Stata).
    To mitigate this problem, we present \benchmark, an automated artifact evaluation benchmark in the domain of software engineering.
    \benchmark has the following advantages:
    (1) It measures LLM agents' ability to generate a code that reproduces the particular claim of the paper; not only the output but the code behind should be submitted.
    (2) It thus has two applications: A script submitted by LLM agents can be used as a partial reproduction scripit that reproduces the particular claim of the paper. Also, in the case that there is a transcription or rounding error in the paper itself, it acts as a credible evidence that this is so.
    (3) It covers diverse software engineering techniques and programming languages.
    Informed by the existing agents' struggle with the \benchmark, we also develop Artisan, an agentic approach for the automated artifact evalutaion.
    Artisan is equipped with many specialized tools to deal with domain-specific problems of artifact evaluation.
    Our evaluation shows that while most LLM agents struggle with \benchmark, Artisan is the most effective LLM agent by producing \artisanpartial{} partial reproduction scripts, outperforming the baseline by \artisanpartialoutperform{}.
    In addition, during our development of \benchmark, we have uncovered \inconsistencies{} inconsistencies between the paper and the artifact, which demonstrates the strength of our approach in benchmark construction.
\end{abstract}
\maketitle

\section{Introduction}
\label{s:introduction}

Since the pilot of artifact evaluation at ESEC/FSE 2011, artifact evaluation has become standard practice in the software engineering and programming languages communities~\cite{DBLP:journals/cacm/KrishnamurthiV15}.
One of its main goals is reproducibility, defined as obtaining similar experimental outcomes with the same experimental setup by a different team~\cite{acm-artifact-badging-v1_1}.
Reproducibility is explicitly cited in Calls for Artifacts and is generally regarded by participants as one of the most important goals of artifact evaluation~\cite{DBLP:conf/sigsoft/Hermann0S20}.
In addition, studies report that artifact evaluation leads to more available, maintained, and documented artifacts~\cite{DBLP:conf/se/0001T0C0H024}.

Despite these benefits, artifact evaluation in its current form is notoriously labor-intensive.
Reports indicate that evaluating a single artifact can take eight to ten hours, often more than reviewing a paper~\cite{DBLP:journals/ieeesp/Hermann22}.
Evidence is mixed on whether papers that undergo artifact evaluation achieve greater visibility: some studies suggest higher visibility~\cite{DBLP:journals/ese/HeumullerNKO20}, while others find no correlation~\cite{DBLP:conf/se/0001T0C0H024}.
Given the significant manual effort, it is unclear whether the benefits to individual researchers outweigh the costs for both authors and artifact users.

\begin{table}[t]
\caption{Comparison of prior benchmarks and our benchmark. ML: Machine Learning. NLP: Natural Language Processing. CS: Computer Science. SS: Social Science. Med: Medicine. SE: Software Engineering.}
\label{t:benchmark_comparison}
\centering
\renewcommand{\arraystretch}{1.25}
\begin{tabularx}{\linewidth}{lXXXX}
\toprule
 & \textbf{SUPER}~\cite{DBLP:conf/emnlp/BoginYG0BCSK24} & \textbf{CORE-Bench}~\cite{DBLP:journals/tmlr/SiegelKNSN24} & \textbf{Repro-Bench}~\cite{DBLP:conf/acl/HuZLWPK25} & \textbf{\benchmark~(Ours)} \\
\midrule
Domain & ML, NLP & CS, SS, Med & SS & SE \\
Programming Languages & Python & Python, R & Stata, R, MATLAB, Julia, Python & Python, Java, Rust, Ocaml, Scala, Bash, C \\
\#Papers & 45 & 90\textsuperscript{1} & 112 & \papersetsize \\
\#Tasks & 45\textsuperscript{2} & 270 & 112 & \tablesetsize \\
\#Inconsistencies Found & 0 & 0 & 0 & 10 \\
Time Limit & 30 minutes & 45 minutes & 2 hours & 8 hours \\
Judging & Accuracy of outputs & Accuracy of outputs & Accuracy of reproducibility scores & Submitted script reproduces the claim \\
\bottomrule
\end{tabularx}


\vspace{2mm}
\footnotesize \textsuperscript{1}\;37 in CS.
\footnotesize \textsuperscript{2}\;Expert set. SUPER additionally provides 152 Masked sets and 604 Auto sets.
\end{table}

To mitigate these costs, recent work benchmarks the performance of LLM agents on reproducing research results.
Table~\ref{t:benchmark_comparison} summarizes these efforts.
SUPER~\cite{DBLP:conf/emnlp/BoginYG0BCSK24} evaluates the ability of LLM agents to reproduce results from ML and NLP papers.
CORE-Bench~\cite{DBLP:journals/tmlr/SiegelKNSN24} has a similar overall goal but covers broader domains (social science, medicine) and includes vision-language tasks.
Repro-Bench~\cite{DBLP:conf/acl/HuZLWPK25} focuses on reproducing social science research and adopts a more realistic problem setup.

However, we identify several limitations in existing benchmarks:
(1) They largely formulate reproduction as an output-prediction problem, which undermines credibility.
SUPER and CORE-Bench measure only the accuracy of the reproduced outcomes; the process by which agents obtain those results is not evaluated.
This diverges from the goals of artifact evaluation in software engineering and is more susceptible to data leakage.
Repro-Bench measures agreement with ground-truth reproducibility scores on a 1–4 scale.
While an ACM artifacts badge (akin to a reproducibility score) is a valuable \emph{signal} produced by artifact evaluation, it is not the main goal.
(2) Due to this problem formulation, their utility is limited.
Unlike human artifact evaluation, which increases confidence that a paper’s results are reproducible, running these benchmarks does not confer comparable confidence.
They can measure agent performance on the posed tasks but cannot augment or replace current artifact evaluation.
(3) Coverage is limited in both software engineering techniques (ML / NLP) and programming languages (Python, R, and Stata).

To address these limitations, we present \benchmark, an automated artifact evaluation benchmark for software engineering.
\benchmark~has the following characteristics:
(1) It measures an agent’s ability to \emph{generate code} that reproduces a specific claim; both the output and the submitted code are graded.
It also distinguishes \emph{fast paths} and \emph{short paths} that agents may take to reproduce results, a crucial distinction not highlighted in prior work.
(2) \benchmark~has two practical applications:
A script submitted by an agent can serve as a push-button reproduction script for the paper’s particular claim.
If the script’s outcome differs from the paper’s, it provides credible evidence that the paper, not the reproduction, contains a transcription or rounding error.
(3) \benchmark~covers diverse software engineering techniques (e.g., Empirical Study, Security, Fuzzing, Static Analysis, Dynamic Analysis) and programming languages (Python, Java, Rust, Ocaml, Scala, Bash, C).

Motivated by agents’ struggles on \benchmark, we also develop \emph{Artisan}, an agentic approach to automated artifact evaluation.
Artisan includes specialized tools for domain-specific challenges in artifact evaluation (Table~\ref{t:artisan_tools}).
\todo{Add technical novelties of Artisan.}

Our evaluation shows that while most LLM agents struggle on \benchmark, Artisan is the most effective, producing \evalgoodscript{} reproduction scripts and outperforming the baseline by \evalgoodscriptoutperform{}.
During the construction of \benchmark, we also uncovered \inconsistenciessize~inconsistencies between papers and artifacts and \fastpathsize~fast paths that artifacts provide, demonstrating the strength of our approach to benchmark construction.

In summary, this paper contributes the following:
\begin{itemize}
    \item \benchmark, a novel benchmark that evaluates the capabilities of AI agents to reproduce Software Engineering papers using script submission.
    \item Artisan, an LLM-based agent which incorporates several tools to aid the reproduction of software engineering research artifacts.
    \item Empirical evaluation which shows that (1) \benchmark~is challenging for the current state of the art LLM agents and (2) Artisan effectively addresses the common pitfalls faced by LLM agents.
    \item We make \benchmark~and Artisan publicly available: https://github.com/doehyunbaek/artisan
\end{itemize}

\section{Background}
\label{s:background}

\subsection{Artifact Evaluation}

\section{Benchmark}
\label{s:benchmark}

\subsection{Overview}

The notable difference between our work and previous work is that we are script-based.
Why is it important?
It is immediately useful. It acts as a standalone script that is push-button runnable which reproduces the particular table.
It is supported by a concrete execution.
What’s a good word to capture this? Realistic? Execution-based? Execution-proof? Evidence-accompanying?
Comparison with existing work in that they formulate reproduction as a prediction problem but we as a code generation problem?
It (loosely) implies that this can be used to correct mistakes made in the paper.
Previous works have not pointed out any errors in the paper.
We don’t need separate ground truth. Numbers in the paper is the ground truth.
Less prone to data combination?
Or it doesn’t matter if it’s contaminated because it’s useful?
What’s the argument for this?

\subsection{Paper Selection}
- Top SE (ICSE, FSE, ASE, ISSTA) with Available and Reusable badges 2024: 114
- Find artifact url
- Packaged using Docker: 77
- Exclude Commercial LLM API use Github token (non-public): 67
- Exclude GPU use: 46
- ess than 8 hours per task (when they mention it): 40

\subsection{Task Selection}
We want our benchmark to be
- No cherry-picking
- Extensive (100~300) but not too expensive and long-running to run.
- Manual validation is desired but should not take too much time.

There are few different options:
Run full task set without modification (162)
Manually validate all task set (162) to filter out unreproducible and keep the partial table
Random sample to smaller size (25 / 50)

I argue for the second option (manually validate all task sets).
It is not too much work (1 hour per task) to check whether it is sensible to reproduce the table by running artisan.
We have about 127 tables to manually validate and we can sensibly validate them by the end of the ISSTA submission
Makes our benchmark higher quality. Less underestimation of agent capability
If we manually validate all without bias, no cherry-picking.
Is using artisan as a time-saving measurement already a bias?

Randomly sample and manually validate and time budget with time budget.

\section{Approach}
\label{s:approach}

% \section{Implementation}
% \label{s:implementation}

\section{Evaluation}
\label{s:evaluation}

\begin{table}[t]
  \caption{Effectiveness of Artisan on the \benchmark}
  \label{t:big_table}
  \begin{tabular}{lccc}
\toprule
 ID            &  Auto Scripts  &  Manual Scripts  &  Total Tables  \\
\midrule
 pythonic      &       8        &        1         &       9        \\
 trace2inv     &       0        &        0         &       7        \\
 action        &       0        &        0         &       6        \\
 flashsyn      &       0        &        0         &       5        \\
 pmsat         &       1        &        0         &       5        \\
 fuzzslice     &       0        &        0         &       5        \\
 roam          &       0        &        0         &       5        \\
 crossover     &       0        &        0         &       4        \\
 bazel         &       0        &        0         &       4        \\
 rust          &       1        &        0         &       4        \\
 provenfix     &       0        &        1         &       4        \\
 npetest       &       0        &        0         &       4        \\
 unimocg       &       4        &        1         &       4        \\
 llm           &       1        &        0         &       3        \\
 ppt4j         &       1        &        0         &       3        \\
 bloat         &       3        &        0         &       3        \\
 axa           &       0        &        2         &       3        \\
 mutation      &       1        &        0         &       2        \\
 dypybench     &       0        &        0         &       2        \\
 goblinupdater &       0        &        0         &       2        \\
 bcia          &       0        &        0         &       2        \\
 lasapp        &       1        &        0         &       2        \\
 neurojit      &       1        &        0         &       2        \\
 pyinder       &       0        &        0         &       2        \\
 dueforce      &       0        &        1         &       2        \\
 sctype        &       0        &        1         &       1        \\
 inference     &       0        &        0         &       1        \\
 urcrat        &       1        &        0         &       1        \\
 Total         &       23       &        7         &       97       \\
\bottomrule
\end{tabular}
\end{table}
% \section{Limitations}
% \label{s:limitations}

\section{Related work}
\label{sec:relatetd}

% \section{Conclusion}
% \label{sec:conclusion}

% \section*{Data Availability}


\bibliographystyle{ACM-Reference-Format}
\bibliography{references}

\end{document}
\endinput
