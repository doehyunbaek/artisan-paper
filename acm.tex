\documentclass[acmsmall,screen,review,anonymous]{acmart}

\usepackage[capitalise,nameinlink]{cleveref}
\usepackage[many]{tcolorbox}
\usepackage{listings}
\usepackage{enumitem}
\usepackage{pifont}
\usepackage{subcaption}
\usepackage{algorithm}
\usepackage{algpseudocode}
\usepackage{booktabs}
\usepackage{tabularx}
\usepackage{multirow}
\usepackage{makecell}
\usepackage{hyperref}

\begin{CCSXML}
<ccs2012>
   <concept>
       <concept_id>10011007.10011006.10011073</concept_id>
       <concept_desc>Software and its engineering~Software maintenance tools</concept_desc>
       <concept_significance>500</concept_significance>
       </concept>
 </ccs2012>
\end{CCSXML}
\ccsdesc[500]{Software and its engineering~Software maintenance tools}
\keywords{Artifact Evaluation, Benchmarks, LLM agents}

\begin{document}

\newcommand\benchmark{SEAE-Bench}


\newcommand\todo[1]{\textcolor{red}{TODO: #1}}
\newcommand\artisanpartial{\textcolor{red}{TODO}}
\newcommand\artisanpartialoutperform{\textcolor{red}{TODO}}
\newcommand\inconsistencies{\textcolor{red}{TODO}}

\title{Artisan: Automated Artifact Evaluation with Agents}
\begin{abstract}
Artifact evaluation has been shown to be a valuablue way to ensure the reproducibility of software artifacts.
However, in its current form, it remains notoriously labor-intensive.
To mitigate this, recent work has introduced benchmarks that evaluate LLM agents on automatically reproducing research results.
We identify several limitations of existing benchmarks:
(1) They largely frame the task as output prediction.
(2) Their usefulness is limited: LLM agents successfully predicting the outcome of research does not raise the confidence that the paper is reproducible.
(3) They cover a narrow domain of software engineering techniques and programming languages.
To address these issues, we present \benchmark, an automated artifact evaluation benchmark for software engineering research.
\benchmark~has the following characteristics:
(1) It measures an agent’s ability to generate code that reproduces a paper’s particular claim; not only outputs but also the script used to produce them are graded.
(2) The submitted script serves as a reproduction script for the particular claim, and when the result of the script does not match with the results of the paper, it acts as a credible evidence of the transcription or rounding errors on the side of the paper.
(3) It spans diverse software engineering techniques and programming languages.
Informed by existing LLM agents’ struggles on \benchmark, we develop Artisan, an agentic approach to automated artifact evaluation.
Artisan includes specialized tools for domain-specific challenges of artifact evaluation.
Our experiments show that, while most LLM agents struggle on \benchmark, Artisan is effective, producing \artisanpartial{} reproduction scripts and outperforming the baseline by \artisanpartialoutperform{}.
In addition, during the development of \benchmark, we uncovered \inconsistencies{} inconsistencies between papers and their artifacts, demonstrating the strength of our approach to benchmark construction.
\end{abstract}
\maketitle

\section{Introduction}
\label{s:introduction}

Since the pilot of artifact evaluation at ESEC/FSE 2011, artifact evaluation has become standard practice in computer science, being adopted in software engineering~\cite{DBLP:conf/sigsoft/Hermann0S20}, programming languages~\cite{DBLP:journals/cacm/KrishnamurthiV15}, security~\cite{DBLP:conf/ccs/OlszewskiLSWKPU23}, and database communities~\cite{DBLP:journals/sigmod/AthanassoulisTA22}.
One of its main goals is reproducibility, defined as obtaining similar experimental outcomes with the same experimental setup by a different team~\cite{acm-artifact-badging-v1_1}.
Reproducibility is generally regarded by participants as one of the most important goals of artifact evaluation~\cite{DBLP:conf/sigsoft/Hermann0S20}.
In addition, studies report that artifact evaluation leads to more available, maintained, and documented artifacts~\cite{DBLP:conf/se/0001T0C0H024}.

Despite these benefits, artifact evaluation in its current form is notoriously labor-intensive.
Reports indicate that evaluating a single artifact can take eight to ten hours, often more than reviewing a paper~\cite{DBLP:journals/ieeesp/Hermann22}.
In addition, the current artifact evaluation results in the \emph{Artifacts Evaluated} badge as an outcome of the process\cite{acm-artifact-badging-v1_1}.
This serves only as a coarse-grained assessment of the reproducibility of the whole artifact and only a one-time assessment of reproducibility at a specific point in time.
In fact, recent work on the reproducibility of artifacts~\cite{DBLP:journals/corr/abs-2510-25506} points out that even artifacts that have been awarded the \emph{Artifacts Evaluated} badge are not reproducible after one year.

\begin{table}[t]
\caption{Comparison of existing approaches and Artisan}
\label{t:agent_comparison}
\centering
\begin{tabularx}{\linewidth}{lXXXX}
\toprule
 & Manual Evaluation  & Rating-based Agents~\cite{reprobench,heye} & Output-based Agents~\cite{super, corebench} & \textbf{\approach~(Ours)} \\
\midrule
Saves Manual Effort           &            & \checkmark & \checkmark & \checkmark \\
Fine-grained Assessment       &            &            & \checkmark & \checkmark \\
Produces Scripts              &            &            &            & \checkmark \\
Detects Inconsistencies       & \checkmark &            &            & \checkmark \\
Detects Fast-Path             &            &            &            & \checkmark \\
\bottomrule
\end{tabularx}
\end{table}

To address this, recent work explores applying LLM agents to the task of automatically reproducing research results.
These approaches can be classified in terms of the output of the LLM agents.
Rating-based agents~\cite{reprobench,heye} output how reproducible the artifact is in terms of a score.
However, the assessment is still coarse-grained; there is no information on which specific claim is reproducible or not.
Output-based agents~\cite{super, corebench} output the numerical results they reproduced, giving a fine-grained assessment of reproducibility.
Still, like rating-based agents, when these agents fail to reproduce research results, it is unclear whether the error lies with the agent or the artifact.
As agents are an emerging technique, it is easier to attribute the error to the agent.
To our knowledge, there is no report of errors in artifacts found by agents while attempting reproduction.

Motivated by the current situation, we ask the question: What if we ask LLM agents to generate code that reproduces research results?
We call this approach \approach, an automated artifact evaluation approach that frames the task as a code generation task.
Given a numerical result from a research paper, \approach~autonomously takes actions to reproduce the result.
To aid the process, we provide a set of tools specialized to improve the domain-specific task of artifact evaluation.
Subsequently, \approach~encapsulates this knowledge into a \emph{\goodscript}, which can be run independently to reproduce research results.
If the results match, our judging mechanism also evaluates \goodscript{}~on how the result is obtained, by using an LLM-as-Judge that evaluates the completeness of the reproduction.
This mechanism precludes trivial~\goodscript s~that merely copy the checked-in research results, which we call \emph{\fastpath}.
If the results mismatch, it could be that the error is on the side of either the paper or the artifact, which we call \emph{\newbug}.
The presence of a \goodscript, which runs independently of the LLM, acts as credible evidence if this is the case.

In addition, \approach~has the following advantages:
(1) It gives a fine-grained reproduction of a specific result.
Each \goodscript~that \approach~produces reproduces only the result that it is tasked with, making it clear which claim is reproduced.
(2) Once produced, \goodscript~can run independently of \approach.
This removes the possibility of agent error in reproduction, opening up the possibility of catching actual errors in the artifacts.
(3) Reproduction script can be run multiple times in the future at a much lower cost.
No manual evaluation or LLM agent is required to rerun the \goodscript.

To evaluate \approach, we develop \benchmark, a new benchmark that evaluates LLM agents' capabilities to generate \goodscript.
\benchmark~comprises 61 tasks derived from 23 Software Engineering papers.
All tasks in \benchmark~are manually validated for reproducibility, and the benchmark offers the following advantages compared to existing ones:
(1) \benchmark~targets the reproduction of tables extracted from research papers as-is.
Only non-deterministic or time-consuming parts are filtered out to get the partial table for the subsets.
This differs from previous benchmarks that handcraft results suitable for reproduction.
(2) \benchmark~covers diverse software engineering techniques (e.g., Empirical Study, Security, Fuzzing, Static Analysis, Dynamic Analysis) and programming languages (Python, Java, Rust, OCaml, Scala, Bash, C).
This differs from previous benchmarks limited in both software engineering techniques (ML / NLP) and programming languages (Python, R, and Stata).
(3) \benchmark~covers reproduction tasks that take significantly longer (eight hours), extending the scope that benchmarks cover.

\begin{table}[t]
\caption{Comparison of prior benchmarks and \benchmark. ML: Machine Learning. NLP: Natural Language Processing. CS: Computer Science. SS: Social Science. Med: Medicine. SE: Software Engineering.}
\label{t:benchmark_comparison}
\centering
\renewcommand{\arraystretch}{1.25}
\begin{tabularx}{\linewidth}{lXXXX}
\toprule
 & SUPER~\cite{super} & CORE-Bench~\cite{corebench} & Repro-Bench~\cite{reprobench} & \textbf{\benchmark~(Ours)} \\
\midrule
Domain & ML, NLP & CS, SS, Med & SS & SE \\
Programming Languages & Python & Python, R & Stata, R, MATLAB, Julia, Python & Python, Java, Rust, Scala, OCaml, Bash \\
Time Limit & 10 minutes & 45 minutes & 2 hours & 8 hours \\
\#Papers & 45 & 90\textsuperscript{1} & 112 & \papersetsize \\
\#Tasks & 45\textsuperscript{2} & 90\textsuperscript{3} & 112 & \tablesetsize \\
\#Inconsistencies Found & 0 & 0 & 0 & \newbugsize \\
\#Fast-path Found & 0 & 0 & 0 & \fastpathsize \\
\bottomrule
\end{tabularx}
\footnotesize \textsuperscript{1}\;37 for CS.
\footnotesize \textsuperscript{2}\;Expert set. super additionally provides 152 Masked sets and 604 Auto sets.
\footnotesize \textsuperscript{3}\;CORE-Bench-Hard. CORE-Bench additionally provides 90 CORE-Bench-Easy and CORE-Bench-Medium.
\end{table}

We evaluate \approach~on \benchmark, showing that \approach~is effective, producing \goodscripttsize{} \goodscript s and outperforming the baseline by \goodscripttsizeoutperform{}$\times$.
In the process, we uncovered \newbugsize~inconsistencies between papers and artifacts and \fastpathsize~fast paths where agents obtain results without genuine reproduction, demonstrating the strength of our approach.

In summary, this paper contributes the following:
\begin{itemize}
    \item Artisan, an LLM agent that automates artifact evaluation through code generation.
    \item \benchmark, a benchmark that evaluates the code generation capabilities of LLM agents to reproduce Software Engineering papers.
    \item Empirical evaluation showing that Artisan generates \goodscripttsize{} new \goodscript{}, outperforming the baseline by \goodscripttsizeoutperform{}$\times$.
    \item Insights that automated artifact evaluation through code generation can reveal paper-artifact inconsistencies and fast-paths taken by agents.
    \item Publicly available Artisan and \benchmark: https://github.com/doehyunbaek/artisan
\end{itemize}

\section{Background}
\label{s:background}

\subsection{Artifact Evaluation}

\subsection{LLM Agents for Software Engineering}


\section{Benchmark}
\label{s:benchmark}

In this section, we present the methodology used to construct \benchmark.
We first state the requirements we target (Section~\ref{s:requirements}), then describe paper selection (Section~\ref{s:paper_select}) and task selection (Section~\ref{s:task_select}).

\subsection{Requirements}
\label{s:requirements}

We impose two core requirements on \benchmark.
(1) Tasks should closely reflect real-world artifact evaluation so that performance on the benchmark generalizes to practical use.
This addresses limitations of prior benchmarks~\cite{super,corebench,reprobench}, which we find do not fully mirror artifact evaluation as practiced in computer science research.
(2) Manual validation is required.
Although SWE-Bench~\cite{jimenez2024swebench} is influential for evaluating LLMs on real software engineering issues, subsequent efforts such as SWE-Bench Verified~\cite{chowdhury2024swebenchverified} and Wang et al.~\cite{DBLP:journals/corr/abs-2503-15223} highlight the importance of human verification to avoid overestimating or underestimating agent capabilities.
These requirements introduce trade-offs.
For example, the task suite must be broad enough to cover diverse use cases yet small enough to permit manual validation.
Further, a very large task set size increases time and monetary costs, which can be prohibitive for resource-constrained academic groups~\cite{DBLP:conf/iclr/ChanCJASMSLMPMW25}.

\subsection{Paper Selection}
\label{s:paper_select}

\begin{table}[t]
  \caption{Paper selection criteria.}
  \label{t:paper_select}
  \centering
  \begin{tabular}{@{}l r@{}}
    \toprule
    Criterion & Count \\
    \midrule
    Flagship SE conference papers (2024) with \emph{Available} and \emph{Reusable} badges & 114 \\
    Packaged using Docker & 72 \\
    No non-public API use & 63 \\
    No GPU use & 44 \\
    Less than eight hours per task & 32 \\
    Manual reproduction successful & \papersetsize \\
    \bottomrule
  \end{tabular}
\end{table}

\begin{table}[t]
  \centering
  \caption{Distribution of Software Engineering Techniques and Programming Languages.}
  \label{t:paper_distribution}
  \begin{tabular}{lclc}
    \toprule
    \textbf{Technique} & \textbf{Papers} & \textbf{Language} & \textbf{Papers} \\
    \midrule
    Empirical Study             & 5 & Python & 13 \\
    Static Analysis             & 5 & Java   & 4  \\
    Security                    & 2 & Rust   & 2  \\
    Dynamic Analysis            & 2 & Scala  & 2  \\
    Software Maintenance        & 2 & OCaml  & 1  \\
    Others                      & 7 & Bash   & 1  \\
    \midrule
    \textbf{Total}              & \textbf{23} & \textbf{Total} & \textbf{23} \\
    \bottomrule
  \end{tabular}
\end{table}

Guided by these requirements, we curate a set of \papersetsize~papers following the process in Table~\ref{t:paper_select}.
(1) We gather 114 papers from four flagship software engineering venues (ICSE, FSE, ASE, ISSTA) in 2024 with ACM \emph{Artifacts Available} and \emph{Artifacts Evaluated—Reusable} badges~\cite{acm-artifact-badging-v1_1}.
We then manually locate links to the accompanying artifacts.
For FSE and ISSTA, the ACM Digital Library lists artifact links; for other venues, we use a web search (e.g., Google) with the paper title.
(2) We exclude 42 papers whose artifacts are not packaged using Docker.
This reflects the fact that Docker and related virtualization technologies are now standard in artifact evaluation~\cite{icse2024-ae,fse2024-ae,ase2024-ae}.
To implement this, we case-insensitively search Markdown and PDF files for the term ``docker''.
% \footnote{\url{https://github.com/doehyunbaek/artisan/blob/bc7e6cd/evaluation/select_papers.py\#L234}}
(3) We exclude nine papers that rely on non-public APIs.
To implement this, we search Python files and Jupyter notebooks for ``openai'' or ``anthropic''.
We also search Markdown files for ``etherscan''.
This excludes seven papers using OpenAI APIs and two papers using Etherscan APIs.
% .\footnote{\url{https://github.com/doehyunbaek/artisan/blob/bc7e6cd/evaluation/select_papers.py\#L276}}
(4) We exclude 19 papers that require GPUs or specialized hardware.
This is to maintain accessibility across hardware setups.
We search for common GPU keywords (e.g., \texttt{cuda}, \texttt{cudnn}, \texttt{tensorflow}, \texttt{nvidia-smi}, \texttt{gpus}) while ignoring non-code extensions (e.g., \texttt{.txt}, \texttt{.csv}, \texttt{.json}, \texttt{.log}).
% \footnote{\url{https://github.com/doehyunbaek/artisan/blob/bc7e6cd/evaluation/select_papers.py\#L312}}
This aligns with previous work~\cite{super} that excludes papers that require GPU use due to cost.
(5) We exclude twelve papers that require more than eight hours to reproduce.
We search the paper and artifact for the keyword ``hour'' and exclude papers that explicitly mention experiment time beyond eight hours.
Out of twelve papers excluded, four papers take more than 12 hours, seven papers take more than 24 hours, and one paper takes more than 125 hours.
This time requirement is much higher than previous work (10 minutes for SUPER~\cite{super}), which might make the evaluation longer.
We motivate this decision by the need to cover diverse software engineering techniques, where many take hours of compute to output meaningful outcomes, even partial ones.
(6) Finally, we exclude nine papers that we could not manually reproduce within an eight-hour budget.
Out of the nine papers excluded, two are excluded because most of the evaluation dataset is missing, two because raw data is missing, two because the script to generate results from raw data is missing, two because the outputs are non-deterministic, and one because the Docker image mentioned in the artifact is no longer accessible.

Table~\ref{t:paper_distribution} shows the distribution of software engineering techniques and programming languages covered by our paper selection.
For software engineering techniques, \benchmark~covers diverse techniques not covered by existing benchmarks.
Papers in the ``Others'' category include one each for Fuzzing, Mutation Testing, Model-based Testing, Mobile Applications, Program Repair, Test Automation, and AI for Software Engineering.
One thing to note is a possible under-representation of AI4SE and SE4AI techniques.
This is mainly due to our exclusion of papers using non-public APIs and GPUs.
Still, there is one paper~\cite{DBLP:conf/kbse/LeeJL24} that makes use of classical machine learning techniques (Random Forest and XGBoost) and does not use either non-public APIs or GPUs.
For programming languages, Python is the most prevalent, comprising 57\% of the papers.
However, there is a sizable representation of other languages, notably Java, Rust, OCaml, Scala, C, and even Bash, which were not covered in previous benchmarks~\cite{super, corebench, reprobench}.

\subsection{Task Selection}
\label{s:task_select}

\begin{table}[t]
  \caption{Task selection criteria.}
  \label{t:task_select}
  \centering
  \begin{tabular}{@{}l r@{}}
    \toprule
    Criterion & Count \\
    \midrule
    Total number of tables & 89 \\
    Exclude non-results & 68 \\
    Exclude missing code & 62 \\
    Exclude non-deterministic only & \tablesetsize \\
    \bottomrule
  \end{tabular}
\end{table}

\begin{table}[t]
  \centering
  \caption{Distribution of inconsistent tables and partial tables.}
  \label{t:task_distribution}
  \begin{tabular}{lclc}
    \toprule
    \textbf{Consistency} & \textbf{Tasks} & \textbf{Original Tables} & \textbf{Tasks} \\
    \midrule
    Consistent Table    & 38 & Original Table  & 41 \\
    Inconsistent Table  & 23 & Partial Table   & 20 \\
    \midrule
    \textbf{Total}              & \textbf{\tablesetsize} & \textbf{Total} & \textbf{\tablesetsize} \\
    \bottomrule
  \end{tabular}
\end{table}

From the \papersetsize~papers selected, we curate \tablesetsize~tasks that are given to LLM agents as a single task instance.
We start with 89 tables contained in the~\papersetsize~papers.
(1) We exclude 21 non-result tables.
Out of the 21 non-result tables excluded, 14 tables contain descriptions of evaluation setups, and 7 tables contain descriptions of the approach.
(2) We exclude six tables that were missing code to reproduce the results.
For all six tables, code reproducing the other parts of the experiment is present, but the code for some parts is missing.
(3) We exclude one table that only contains non-deterministic results.
For PPT4J~\cite{DBLP:conf/icse/Pan00Z0024}, Table 3, the table reports only the time consumption of the approaches.
As~\approach~does not handle non-deterministic results such as time consumption (Section~\ref{s:limitations}), we exclude this table.
For the remaining~\tablesetsize~tasks, we reproduce the tables and obtain ground truth scripts that reproduce the table.

Table~\ref{t:task_distribution} shows whether the tables used for the tasks are consistent or partial with respect to the tables in the paper.
23 tables that we use for the benchmark are inconsistent with the table in the paper.
For 19 tasks, this is due to~\newbug~that we found, which we discuss in detail in Section~\ref{s:inconsistencies}.
For three, it is because the paper authors claim the results of the artifact could be slightly different from those in the paper~\cite{DBLP:journals/pacmse/Song0LCR24}, and for one, it is because the implementation used for the paper~\cite{DBLP:conf/kbse/Bock0C24} contains a bug, which is fixed now.
20 tables that we use for the benchmark contain partial entries compared to the table in the paper.
For 15, we could only partially reproduce the results of the original table with our ground truth scripts.
For 5, we exclude the parts where the output is non-deterministic.


% \textbf{Ground Truth Kind} & \textbf{Tasks} \\
% & Full Reproduction       & 31 \\
% & Downstream Reproduction & 30 \\

% \begin{figure}[t]
%   \centering
%   \begin{minted}{bash}
% #!/usr/bin/bash
% # Section 1: Write the expected table to /workspace/expected.md
% cat > /workspace/expected.md << 'EOTABLE'
% **Table 2. Ping-pong server: dominant reachable states, glitches, and frequency statistics.**

% | (n) | (n_{reach}) | # Glitches | Mean (\delta_g) fr. | Max (\delta_g) fr. | Min (\delta) fr. |
% | --: | ----------: | ---------: | ------------------: | -----------------: | ---------------: |
% |   3 |           ? |         ?? |                ?.?? |                 ?? |               ?? |
% |   4 |           ? |          ? |                ?.?? |                  ? |               ?? |

% EOTABLE
% # Section 2: Download and extract the artifact
% artisan get https://zenodo.org/records/10423670
% # Section 3: Run the commands to reproduce the results
% # Use docker-compose to run the pmsat container, ensure numpy<2 for compatibility, run mining and parse to produce /workspace/repro.txt
% docker-compose -f pmsat-inference-and-publication-artifacts/pmsat-inference/docker-compose.yml run --rm pmsat /bin/bash -c "cd /pmsat-inference && python -m pip install 'numpy<2' && python run_pmsat_on_traces.py examples-results/ping_pong_example/info.json -nmax 7 && MPLBACKEND=Agg python parse_single_run_results.py TRACE-results/92e710ef352c4739cd7569794272588a" > /workspace/repro.txt
% # Section 4: Format the result into the expected table with artisan format and surround with the required <artisan_submit> block
% echo '<artisan_submit>'
% artisan format --expected /workspace/expected.md --repro /workspace/repro.txt
% echo '</artisan_submit>'
%   \end{minted}
%   \caption{Example~\goodscript~generated by~\approach.}
%   \Description{Example~\goodscript~generated by~\approach.}
%   \label{f:good_script}
% \end{figure}

% \section{Background}
% \label{s:background}

% \subsection{Artifact Evaluation}

% \subsection{LLM Agents for Software Engineering}

\section{Approach}
\label{s:approach}

We first give a brief overview of the approach (Section ~\ref{s:overview}).
Then, we give a detailed explanation of the three tools that we provide to the agent (Section \ref{s:get}, \ref{s:format}, \ref{s:judge})

\subsection{Overview}
\label{s:overview}

\begin{figure}[t]
  \centering
  \begin{minted}{bash}
#!/usr/bin/bash
# Section 1: Artifact download
artisan get {{ artifact_url }}
# Section 2: Reproduction commands (populate from reviewed steps)
<Commands to reproduce the table>
# Section 3: Formatting and submission block
artisan format --expected /workspace/expected.md --repro /workspace/repro.txt
  \end{minted}
  \caption{Script template provided to the agent to guide the code generation.}
  \Description{Script template provided to the agent to guide the code generation.}
  \label{f:script_template}
\end{figure}

\begin{figure}[t]
  \centering
  \begin{minted}{bash}
cat > /workspace/repro_{{ artifact_name }}_table_{{ table_index }}.sh <<'EOF'
{{ script_template }}
EOF
artisan judge /workspace/repro_{{ artifact_name }}_table_{{ table_index }}.sh
  \end{minted}
  \caption{Comand template provided to the agent to guide the script submission.}
  \Description{Comand template provided to the agent to guide the script submission.}
  \label{f:command_template}
\end{figure}

As \approach~tries to automate the artifact evaluation by tasking LLM agents to generate~\goodscript{}, we provide a script template that agents should follow for effective research reproduction.
Figure \ref{f:script_template} shows a script template we provide.
First, agents should fetch the artifact from different artifact repositories.
We provide Get tool (Section~\ref{s:get}) to provide a unified interface to the artifact repositories.
Then, the agent autonomously exercises the fetched artifact to reproduce the research results.
As the commands needed to reproduce the results vary from artifact to artifact, we ask the agent to review the commands it has executed to fill in the necessary commands.
Lastly, the agent has to analyze and format the execution logs into the expected format.
To support this, we provide Format tool (Section~\ref{s:format}), which is an LLM-based formatter that takes an expected table and execution log as an input and output a formatted table.

To guide the agent in reproducing the research result, we allow agent to get a feedback on the \goodscript~they have generated with Judge tool (Section~\ref{s:judge}).
Figure \ref{f:command_template} shows a command template we provide to the agent.
By writing the generated script to a file and calling a Judge tool, the agent can get feedback whether (1) the output of the generated code matches the actual table, and (2) the way~\goodscript~generates the output is appropriate.
The agent can incorporate the feedback if any of the two conditions are not met, guiding the reproduction process.

\subsection{\emph{Get} tool: Fetching Research Artifacts}
\label{s:get}

There currently exists multiple different repository providers that researchers use to archive their artifacts, including Zenodo, Figshare, and GitHub.
Without the specizlied tool, LLM agent can still retrive the artifact from the given tool by utilizing CLI tools like curl to retrieve the artifact.
However, we found that the agent stumbles on interacting with the repository API initially.
In addition, complex probing of the repository API makes the code generation task more complicated, leading to more errors.

To mitigate this, we implement Get tool, which agent can invoke with just the url to fetch the research artifact.
It handles the repository-specific logic silently, downloads the research artifact, extracts the research artifact, and locate the README file of the artifact.
It also supports caching of the artifact that has been already fetched before.
While simple, we found that this is an effective strategy that helps the agent in initial fetching and also generating~\goodscript.

\subsection{\emph{Format} tool: Formatting Execution Outputs into Results}
\label{s:format}

% \begin{figure}[t]
%   \centering
%   \begin{minted}{markdown}
% You format results into GitHub-flavored Markdown tables.

% CRITICAL TEMPLATE RULE:
% Return a table that is IDENTICAL to the provided expected table except that
% every placeholder question mark (?) is replaced by exactly one decimal digit
% (0-9). One ? => one digit. A pattern like "??.??" means two digits, a dot,
% two digits. A pattern like "?,???,???" means digits in those exact slots with
% commas preserved. Do NOT add, remove, or move pipes |, spaces, commas, dots,
% percent signs %, header separators, or rows. Do NOT alter any cell that
% contains no ? characters. Preserve capitalization and ordering exactly.

% DATA FILLING:
% Derive each replaced digit sequence from ONLY the reproduction text.
%   \end{minted}
%   \caption{Part of the system prompt used to implement the Format tool}
%   \Description{Part of the system prompt used to implement the Format tool}
%   \label{f:format}
% \end{figure}

\begin{figure}[t]
  \centering
  % First Subfigure
  \begin{subfigure}[t]{0.48\textwidth}
    \begin{minted}[fontsize=\scriptsize, breaklines=true]{markdown}
**Table 3: Evaluation Results**

| Project Name | Summary   | Annotations | Warnings |
| ---------------------    |             | -------- |
| MarginSwap   | (omitted) |           ? |        ? |
| PoolTogehter | (omitted) |           ? |        ? |
| Tracer       | (omitted) |           ? |        ? |
| Yield Micro  | (omitted) |           ? |        ? |

    \end{minted}
    \caption{Part of the obfuscated table provided as input.}
    \Description{Part of the obfuscated table provided as input.}
    \label{f:format_input_table}
  \end{subfigure}
  \hfill % Adds space between the figures
  % Second Subfigure
  \begin{subfigure}[t]{0.48\textwidth}
    \begin{minted}[fontsize=\scriptsize, breaklines=true]{bash}
Switched global version to 0.8.3
'solc --version' running
Reference: ScType
Error with TMP_57 in function updateHourlyBondAmount
Error with: TMP_57 in function updateHourlyBondAmount
Annotation count: 6
Function count: 20
Executing Group 1
[*] Tested 1 warning for MarginSwap
    \end{minted}
    \caption{Part of the execution output provided as input.}
    \Description{Part of the execution output provided as input.}
    \label{f:format_input_log}
  \end{subfigure}
  \caption{Example input given to the Format tool.}
  \label{f:format_input}
\end{figure}

Even after an agent has successfully reproduced research results by running a series of command, the outputs of the commands are usually not exactly the same as in what is presented in the paper.
Some artifacts also provide scripts that processes these execution outputs into one that is used in the paper, but many artifacts do not provide scripts.
Of course, an LLM agent can try to write the missing processing script itself, but we found that it struggles to write such scripts in a lot of cases.
In many cases, an LLM successfully ran commands necessary to reproduce the research results, but could not get correct score from the judge as the formatting of the execution outputs was inadequate.

To address this, we implement Format tool, which is an LLM-based formatter that takes an expected table and execution log as an input and output a formatted table.
% Figure \ref{f:format} shows a part of the system prompt that we use to implement the Format tool and 
Figure \ref{f:format_input} shows an example input that is passed to a Format tool.
You can see that while table expects the research result in a tabular format (Figure \ref{f:format_input_table}), the output of the execution is just a log file (Figure \ref{f:format_input_log}), and there is not much structure.
In addition, the log file is 1,103 lines in total.
Format tool passes expected table and execution output as inputs with some few-shot examples to the LLM, and outputs the tables with expected values filled-in.

\subsection{\emph{Judge} tool: Evaluating Reproduction Attempts}
\label{s:judge}

In previous work which develops output-based agents~\cite{super, corebench}, we found two characteristics that we found inadequate in our problem formulation:
(1) The agents do not get feedback on their submission while they are working on the problem.
This becomes problematic in our case, as generating code that reproduces the result is more complex problem and has more room for error than outputting just the result.
(2) How the agents obtain the output is not judged.
We observe many~\fastpath~that agents take while attempting to reproduce the result (Section \ref{s:fastpath}), and judge proper consideration of this phenoenon is important for accurate assessment of the LLM agents' capabilities.
Thus, we propose two-tier judging system which addresses both problem, by execution-based judging of the submission~\ref{s:judge_execution}~and LLM-based judging of the reproduction method~\ref{s:judge_llm}.
\subsubsection{Execution-Based Judging of Reproduction Scripts}
\label{s:judge_execution}

In our approach, LLM agents generate and submit a shell script that runs independently of any agent-related machinaries.
Given the submission, our execution-based judging mechanism runs the submitted script in fresh container environment.
Judging mechanism classify three cases where the submitted script fails judging in this phase.
(1) Static Error: when the submitted script fails to run because of syntax error, judging mechanism judge the script as a static error and give the error message as a feedback.
(2) Runtime Error: when the submitted script encounters error while running, judging mechanism judge the script as a runtime error and return which command caused error with what error message.
(3) Result Mismatch: when the submitted script runs to completion without any error but the result is different with the expected values, judging mechanism judge the script as a result mismatch.

\begin{figure}[t]
  \centering
  \begin{subfigure}[t]{0.30\textwidth}
    \begin{minted}[fontsize=\scriptsize, breaklines=true]{markdown}
|   | Calls      | Count     |
|---|:-----------|----------:|
|   | Resolved   | 7,799,929 |
|   | Unresolved |   260,249 |
    \end{minted}
    \caption{Expected Results}
    \Description{Expected Results}
    \label{f:judge_execution_example_1}
  \end{subfigure}
  \hfill
  \begin{subfigure}[t]{0.30\textwidth}
    \begin{minted}[fontsize=\scriptsize, breaklines=true]{markdown}
|   | Calls      | Count     |
|---|:-----------|----------:|
|   | Resolved   | 7,799,929 |
|   | Unresolved |   168,482 |
    \end{minted}
    \caption{Results of the Submitted~\goodscript}
    \Description{Results of the Submitted~\goodscript}
    \label{f:judge_execution_example_2}
  \end{subfigure}
  \hfill
  \begin{subfigure}[t]{0.30\textwidth}
    \begin{minted}[fontsize=\scriptsize, breaklines=true]{markdown}
|   | Calls      | Count     |
|---|:-----------|----------:|
|   | Resolved   | 7,799,929 |
|   | Unresolved |   ???,??? |
Result mismatch: some cells differ
    \end{minted}
    \caption{Feedback Given to the Agents}
    \Description{Feedback Given to the Agents}
    \label{f:judge_execution_example_3}
  \end{subfigure}
  \caption{Example of the Result Mismatch Feedback}
  \label{f:judge_execution_example}
\end{figure}

Importantly, judging mechanism reveals which parts of the result the agent got right, but not the parts it got wrong.
Figure~\ref{f:judge_execution_example}~gives example of this feedback mechanism.
Suppose Figure~\ref{f:judge_execution_example_1}~is the expected result of the reproduction, and Figure~\ref{f:judge_execution_example_2}~is the result obtained after running the submitted ~\goodscript.
As results match for Resolved Count but not for Unresolved Count, Figure~\ref{f:judge_execution_example_3}~informs the agent that the result matches for the Resolved Count but not for Unresolved Count.
This guides the agents for the successful parts but not give away the expected the results for the unsuccessful parts.

In addition, we allow the agents to submit and get the feedback from the judging mechanism while they are still running, different from previous work~\cite{super, corebench}.
This is because:
(1) There is more room for error in the code generation task; we do not want to punish the agents for generating slightly wrong~\goodscript~although it successfully executed commands to reproduce the results
(2) Our judging mechanism does not allow the agents to gain hints on the expected results to guess the results.

\subsubsection{LLM-Based Judging of Reproduction Method}
\label{s:judge_llm}

Even if the submitted~\goodscript~produces the results that match the expected results, it is still possible that no real reproduction is performed and results are obtained through inappropriate means.
Thus, our second phase of judging adopts an LLM-as-Judge to grade how the reproduction is conducted.
Judging mechanism classify three categories of the reproduction method:
(1) \copyrepro{}: The submitted script just copies the results without actually reproducing.
(2) \downrepro{}: The submitted script performs lightweight reproduction using the checked in raw data.
(3) \fullrepro{}: The submitted script perform the full reproduction possible.

Figure~\ref{f:judge_llm_example}~gives example of each variants.
In Figure~\ref{f:judge_llm_example_1}, the script directly copies the checked in result for submission.
Our judging mechanism judges that this script is~\copyrepro, as there is no actual reproduction performed.
Warning is also given to the agent that no credit will be given with this submission.
In Figure~\ref{f:judge_llm_example_2}, the script uses checked-in raw data to process it to generate results.
Although this still fall short of the full reproduction, we judge this to be a still valid reproduction.
This is because the full reproduction of the given experiment takes few hundred hours and checking if the analysis result of the checked-in raw data match the paper result is still valuable.
In Figure~\ref{f:judge_llm_example_3}, the script does the full reproduction from raw data generation to analysis on one example.
This is the ideal case, especially if it can be achieved within a reasonable time.

\begin{figure}[t]
  \centering
  \begin{subfigure}[t]{1\textwidth}
    \begin{minted}[fontsize=\scriptsize, breaklines=true]{bash}
# Section 3: Formatting and submission block
cat artifact/InvCon+/results.txt > /workspace/repro.txt
    \end{minted}
    \caption{Example~\copyrepro{}~scripts.}
    \Description{Example~\copyrepro{}~scripts.}
    \label{f:judge_llm_example_1}
  \end{subfigure}
  \hfill
  \begin{subfigure}[t]{1\textwidth}
    \begin{minted}[fontsize=\scriptsize, breaklines=true]{bash}
# Section 3: Formatting and submission block
docker-compose run pmsat python parse_all_results_timeouts.py benchmarkingset-rc2-results > /workspace/repro.txt
    \end{minted}
    \caption{Example~\downrepro{}~scripts.}
    \Description{Example~\downrepro{}~scripts.}
    \label{f:judge_llm_example_2}
  \end{subfigure}
  \hfill
  \begin{subfigure}[t]{1\textwidth}
    \begin{minted}[fontsize=\scriptsize, breaklines=true]{bash}
# Section 3: Formatting and submission block
docker-compose run pmsat /bin/bash -c "cd /pmsat-inference && python -m pip install 'numpy<2' && python run_pmsat_on_traces.py examples-results/ping_pong_example/info.json -nmax 7 && MPLBACKEND=Agg python parse_single_run_results.py TRACE-results/92e710ef352c4739cd7569794272588a" > /workspace/repro.txt
    \end{minted}
    \caption{Example~\fullrepro{}~scripts.}
    \Description{Example~\fullrepro{}~scripts.}
    \label{f:judge_llm_example_3}
  \end{subfigure}
  \caption{Example of the reproduction mechanisms.}
  \label{f:judge_llm_example}
\end{figure}

\section{Implementation}
\label{s:implementation}

\paragraph{Agent Framework and LLM Model Choice}

We use mini-swe-agent~\cite{minisweagent}, version 1.17.1 to build \approach.
Although mini-swe-agent supports any models supported by LiteLLM~\cite{litellm}, we mainly experimented with gpt-5-mini-2025-08-07 for our day-to-day experimentation and include gpt-5.1-2025-11-13 and gpt-5-nano-2025-08-07 for full-scale evaluation.

\paragraph{Separating Judge tool and Submit tool}

Initially, the agents were given the judge tool to self-grade the submissions themselves.
This led to undesirable results as agents could access the expected results themselves, leaking the results.
Thus, in our implementation, we use a client-server architecture that the agents use Submit tool to submit the script to the server, whereas the Judge tool is the server that receives the incoming submitted script, grades the submission following the method described in Section~\ref{s:judge}, and returns the results to the Submit tool.
This prevents the leakage issue.

\section{Evaluation}
\label{s:evaluation}

To evaluate our approach, we answer the following research questions:

\begin{itemize}[labelindent=\parindent,leftmargin=*]
\item \textbf{RQ1. Effectiveness}:
How effective is \approach{} at generating reproduction scripts?
\item \textbf{RQ2. Efficiency}:
How efficient is \approach{} in terms of time and monetary cost?
\item \textbf{RQ3. Ablation Study}:
What is the impact of different tools in different configurations?
\item \textbf{RQ4. Fidelity and Accuracy of LLM-Based Tools}:
What is the fidelity of the LLM-based Format tool and the accuracy of the LLM-based Judge tool?
\end{itemize}

\subsection{Experimental Setup}

\subsubsection{Datasets}

For RQ1, RQ2, and RQ3, we use \benchmark{} introduced in Section~\ref{s:benchmark} to evaluate our approach and the baselines.
For RQ4-1, we use 60 ground truth scripts to prepare inputs used to evaluate the fidelity of the Format tool.
For RQ4-2, we use 60 ground truth scripts (30~\fullrepro~scripts, 30~\downrepro~scripts), plus 25~\copyrepro~scripts that we collected during development, to evaluate the accuracy of the Judge tool.

\subsubsection{Baselines}

As baselines, we select mini-swe-agent~\cite{minisweagent} and SWE-agent~\cite{sweagent}.
We choose mini-swe-agent for two reasons:
(1) \approach{} is built on top of mini-swe-agent; thus, comparing \approach{} with mini-swe-agent enables a thorough empirical study of \approach{}'s contributions.
(2) mini-swe-agent achieves competitive performance (74.4\% resolved on SWE-Bench Verified with Claude Opus 4.5, compared to 78.8\% for the top score on the current leaderboard), while remaining simple (138~LOC in the core).
We also include SWE-agent as an established state-of-the-art approach for general software engineering tasks.
For both baselines and~\approach, we experiment with gpt-5.1-2025-11-13, gpt-5-mini-2025-08-07, and gpt-5-nano-2025-08-07 as an LLM model.

\subsubsection{Metrics}

To evaluate effectiveness, we measure the sum of~\fullrepro~scripts and~\downrepro{}~scripts each approach generates.
For the reproduction method we report, we report the result of the manual validation, not the one reported by the Judge tool (Section \ref{s:judge}).
To evaluate efficiency, we report total wall-clock time and LLM token cost per task.
To evaluate the fidelity of the Format tool, we measure what percentage of formatted output matches the expected output with the execution log we manually validated to contain all the expected results given as an input.
To evaluate the accuracy of the Judge tool, we measure what percentage of judgement of reproduction matches with the ground truth reproduction method label we have.
We repeat RQ4 experiments ten times to account for an LLM non-determinism.
\subsection{RQ1: Effectiveness}
\label{s:effectiveness}

\begin{table}[t]
  \caption{Effectiveness of \approach{} on \benchmark{}. GPT-5.1 used for model.}
  \label{t:big_table}
  \begin{tabular}{lccccccc}
\toprule
\multirow{2}{*}{ID} & \multicolumn{2}{c}{Success} & \multicolumn{4}{c}{Failure} & \multirow{2}{*}{\shortstack{Total\\Tasks}} \\
\cmidrule(lr){2-3} \cmidrule(lr){4-7}
 & Full Rep. & Down. Rep. & Copy Res. & Res. Mis. & Run. Err. & Stat. Err. & \\
\midrule
pythonic      & & & & & & & 9 \\
action        & & & & & & & 6 \\
pmsat         & & & & & & & 5 \\
roam          & & & & & & & 5 \\
rust          & & & & & & & 4 \\
unimocg       & & & & & & & 4 \\
crossover     & & & & & & & 3 \\
bloat         & & & & & & & 3 \\
provenfix     & & & & & & & 3 \\
axa           & & & & & & & 3 \\
npetest       & & & & & & & 3 \\
llm           & & & & & & & 2 \\
bazel         & & & & & & & 2 \\
dypybench     & & & & & & & 2 \\
goblinupdater & & & & & & & 2 \\
bcia          & & & & & & & 2 \\
lasapp        & & & & & & & 2 \\
neurojit      & & & & & & & 2 \\
mutation      & & & & & & & 1 \\
ppt4j         & & & & & & & 1 \\
sctype        & & & & & & & 1 \\
inference     & & & & & & & 1 \\
urcrat        & & & & & & & 1 \\
\midrule
Total         & & & & & & & 60 \\
\bottomrule
\end{tabular}
\end{table}

\begin{table}[t]
  \caption{Comparison of \approach{} with baselines.}
  \label{t:Baseline}
  \begin{tabular}{lccccccc}
\toprule
\multirow{2}{*}{ID} & \multicolumn{2}{c}{Success} & \multicolumn{4}{c}{Failure} & \multirow{2}{*}{\shortstack{Cost}} \\
\cmidrule(lr){2-3} \cmidrule(lr){4-7}
 & Full Rep. & Down. Rep. & Copy Res. & Mis. Res. & Run. Err. & Stat. Err. & \\
\midrule

mini-swe-agent & & & & & & & \\
\shortstack{-- w/ gpt-5-nano} & 0 & 0 & 0 &  0 &  0 & 60 & \$0.008 \\
\shortstack{-- w/ gpt-5-mini} & 0 & 0 & 0 & 44 & 14 &  2 & \$0.061 \\
\shortstack{-- w/ gpt-5.1}    &  & & & & & & \\

swe-agent & & & & & & & \\
\shortstack{-- w/ gpt-5-nano} &  0 & 1 & 0 & 34 & 17 & 8 & \$0.061 \\
\shortstack{-- w/ gpt-5-mini} & 34 & 0 & 0 & 23 &  4 & 0 & \$0.136 \\
\shortstack{-- w/ gpt-5.1}   & & & & & & & \\
\midrule

Artisan & & & & & & & \\
\shortstack{-- w/ gpt-5-nano} & & & & & & & \\
\shortstack{-- w/ gpt-5-mini} & & & & & & & \\
\shortstack{-- w/ gpt-5.1}   & & & & & & & \\
\midrule

\bottomrule
\end{tabular}

\end{table}

\subsection{RQ2: Efficiency}
\label{s:efficiency}

\subsection{RQ3: Ablation Study}
\label{s:ablation}

\begin{table}[t]
  \caption{Ablation study of \approach. GPT-5.1 used for all experiments.}
  \label{t:ablation}
  \begin{tabular}{lccccccc}
\toprule
\multirow{2}{*}{ID} & \multicolumn{2}{c}{Success} & \multicolumn{4}{c}{Failure} & \multirow{2}{*}{\shortstack{Total\\Tasks}} \\
\cmidrule(lr){2-3} \cmidrule(lr){4-7}
 & Full Rep. & Down. Rep. & Copy Res. & Res. Mis. & Run. Err. & Stat. Err. & \\
\midrule
mini-swe-agent      & & & & & & & 61 \\
w/o Judge      & & & & & & & 61 \\
w/o Format      & & & & & & & 61 \\
w/o Get      & & & & & & & 61 \\
artisan      & & & & & & & 61 \\
\bottomrule
\end{tabular}
\end{table}

\subsection{RQ4: Fidelity and Accuracy of LLM-Based Tools}
\label{s:llmtools}

\subsubsection{RQ4-1: Fidelity of the Format tool}
\label{s:eval_format}

\subsubsection{RQ4-2: Accuracy of the Judge tool}
\label{s:eval_judge}

\section{Discussion}
\label{s:discussion}

\subsection{Inconsistencies Found Between Paper and Artifact}
\label{s:inconsistencies}

\begin{table}[t]
  \caption{Inconsistencies Found Between Paper and Artifact. \textsuperscript{\textdagger} denotes confirmatioin by the authors.}
  \label{t:inconsistencies}
  \begin{tabular}{l r r r r}
\toprule
\textbf{Table File} & \textbf{\# Diff Cells} & \textbf{Paper Value} & \textbf{Artifact Value} &\textbf{Reason} \\
\midrule
action\_table\_1.md & 4 & 0.30 & 0.32 & Paper Error \\
action\_table\_2.md & 29 & 33.8 & 36.2 & Paper Error \\
action\_table\_3.md & 1 & 66.3 & 66.4 & Rounding Error \\
action\_table\_4.md & 21 & -55.14 & -62.63 & Paper Error \\
action\_table\_5.md & 5 & -10.41 & -99.78 & Paper Error \\
axa\_table\_1.md & 2 & 863 & 848 & Paper Error \\
bazel\_table\_5.md & 4 & 28 & 29 & Paper Error \\
dypybench\_table\_3.md & 1 & 1701 & 1543 & Artifact Error\textsuperscript{\textdagger}\textsuperscript{1} \\
interference\_table\_2.md & 4 & $8.13 \times 10^{285960}$ & $8.58 \times 10^{506}$ & Paper Error \\
lasapp\_table\_1.md & 1 & 1549 & 1553 & Artifact Updated After Paper\textsuperscript{\textdagger} \\
llm\_table\_3.md & 4 & 1.15 & 0.15 & Paper Error \\
npetest\_table\_2.md & 27 & 100 & 68 &  Data Difference\textsuperscript{\textdagger} \\
npetest\_table\_3.md & 2 & 73 & 74 &  Data Difference\textsuperscript{\textdagger} \\
pmsat\_table\_3.md & 1 & 0.6 & 0.5 & Rounding Error \\
pmsat\_table\_5.md & 1 & 17.33 & 17.3 & Rounding Error \\
pythonic\_table\_3.md & 5 & 0.14 & 0.24 & Paper Error \\
pythonic\_table\_4.md & 1 & 0.15 & 0.16 & Rounding Error \\
pythonic\_table\_9.md & 1 & 0.04 & 0.05 & Rounding Error \\
rust\_table\_1.md & 6 & 98.88\% & 98.93\% & Data Difference\textsuperscript{\textdagger}\\
sctype\_table\_3.md & 2 & 1 & 2 & Paper Error \\
urcrat\_table\_1.md & 1 & 72199 & 72201 & Data Difference\textsuperscript{\textdagger}\\
\bottomrule
\end{tabular}
\footnotesize \textsuperscript{1}Fixed by https://github.com/sola-st/DyPyBench/commit/bccfda7
\end{table}

During the development of~\approach, we have encountered~\newbugsize~cases where there was some errors in papers or artifacts, which we encountered through numerical results being output by artifacts being inconsistent with the papers, a phenomena we term~\newbug.
To be clear, we do not classify cases where artifacts themselves warn about the possible inconsistencies as~\newbug; thus all~\newbugsize~we report are newly discovered errors that have not been disclosed publicly before.
Table~\ref{t:inconsistencies} summarizes the~\newbug~we found.
Overall, we found~\newbugsize~\newbug~from 13 papers.
This accounts for 22\% (13 / 60) of the papers we applied~\approach.
Note that all of the 60 papers that we apply~\approach~underwent an artifact evaluation process in the major SE conferences, and got awarded Artifacts Evaluated - Reusable badge.
Still, our approach could find inconsistencies in up to 29 entries in a table, and a value different up to $8.13 \times 10^{285960}$.

To give more insight into the~\newbug~we found, we perform a manual analysis to find out the reasons for the inconsistencies.
We come out with five different categories of reasons for~\newbug.
% Rounding Error
5 /~\newbugsize~belong to Rounding Error, which we attribute when the inconsistencies found could be explained away by the wrong use of the rounding algorithm.
For the rest of the Inconsistencies, we reported our analysis to the paper authors.
12 / 16 of the reported inconsistencies received confirmations by the authors.
% Data Difference
4 / 16 belong to Data Difference, which we attribute when the data used in the paper is different from the data present in the artifact.
For all four such cases, the data used in the paper is lost, rendering the faithful reproduction of the paper results unlikely.
This highlights the need for automated artifact evaluation technique like~\approach, which could have prevented such cases had it been applied earlier.
% Artifact Error
3 / 16 belong to Artifact Error, which we attribute when we could find apparent errors in the artifact data or code.
Two out of three artifact errors are already fixed and the authors shared an intention to fix the remaining one.
\footnote{For Artifact Error of lasapp\cite{lasapp}, both the number of the paper and the artifact used for the paper results were buggy. We applied~\approach~to the fixed version of the artifact as the buggy version was not accessible to the public, which led to the disclosure of the buggy version.}
% Transcription Error
1 / 16 belong to Transcription Error, which we attribute when the inconsistencies likely stemed from manual process of transcribing artifact values to the paper.
% Paper Error
8 / 16 belong to Paper Error, which is a general catch-all category of all inconsistencies attributable to error in the paper.
\todo{more analysis on 8 paper errrors.}

\subsection{Fast Paths Taken by Agents}
\label{s:fastpath}

\begin{table}[t]
    \centering
    \caption{Frequency of Fast Paths Taken by Agents}
    \label{t:fastpath}
    \begin{tabular}{cccccc}
        \toprule
         \textbf{Checked-in Results} & \textbf{Checked-in Papers}  & \textbf{Notebook Outputs} & \textbf{Paper Texts} & \textbf{Others} & \textbf{Total} \\
        \midrule
        12 & 7 & 7 & 3 & 1 & \fastpathsize \\
        \bottomrule
    \end{tabular}
\end{table}

% action_table_2.log: Notebook Outputs
% action_table_3.log: Notebook Outputs
% action_table_5.log: Notebook Outputs
% action_table_6.log: Notebook Outputs
% baro_table_2.log: Checked-in Papers
% bloat_table_3.log: Other (Paper Texts)
% buggyui_table_2.log: Checked-in Papers
% dcm_table_3.log: Checked-in Papers
% fuzzslice_table_1.log: Checked-in Papers
% gradual_table_3.log: Checked-in Results
% hypertesting_table_1.log: Checked-in Results
% lasapp_table_2.log: Other (Paper Texts)
% llm_table_1.log: Notebook Outputs
% llm_table_2.log: Checked-in Papers
% llm_table_3.log: Notebook Outputs
% mwt_table_2.log: Other (download from arxiv)
% newton_table_4.log: Notebook Outputs
% provenfix_table_2.log: Checked-in Results
% provenfix_table_3.log: Checked-in Results
% rca_table_3.log: Checked-in Papers
% reclues_table_4.log: Checked-in Papers
% rust_table_1.log: Checked-in Results
% rust_table_5.log: Other (Paper Texts)
% sugar_table_2.log: Checked-in Results
% trace2inv_table_7.log: Checked-in Results
% triad_table_3.log: Checked-in Results
% unimocg_table_1.log: Checked-in Results
% unimocg_table_2.log: Checked-in Results
% unimocg_table_3.log: Checked-in Results
% unimocg_table_4.log: Checked-in Results

During the development of~\approach, we have encountered~\fastpathsize~cases where agents generate~\goodscript~that merely copies checked-in results as a method of ``reproduction'', a phenomena we term~\fastpath.
Although similar misalignment issues are reported in other domains~\todo{cite}, to our knowledge, we are the first to report a misalignment issue of this kind in he domain or artifact evaluation.

We perform a manual analysis of trajectories of agents that produced these~\fastpath, and report the findings in Table~\ref{t:fastpath}.
For 12 / 30 cases, artifacts contain some checked-in research results (Checked-in Results).
Agents discover them while exploring the artifact and directly copy these results.
For 7 / 30 cases, artifacts contain papers in pdf forms (Checked-in Papers).
Agents convert the pdfs into a text-readable form and directly copy the expected tables.
For 7 / 30 cases, artifacts contain Jupyter Notebooks which contain the output of the previous run, which agents directly copy (Notebook Outputs).
For 3 / 30 cases, although we give a obfuscated table when we provide the paper, agents still try to read the texts surrounding the obfuscated table and infer the expected results (Paper Texts).
Lastly, on one case, after other attempts fail, the agent directly downloads the paper by searching arxiv.org with the paper title and obtains the preprint.
These various ways of obtaining research results while actually not doing the reproduction shows a need for verification of not only the output but the method of reproduction, as done by~\approach.

\subsection{Threats To Validity}
\label{s:threats}

\subsection{Limitations}
\label{s:limitations}

\section{Related work}
\label{sec:relatetd}

\paragraph{Artifact Evaluation}

\paragraph{Research Reproduction Automation}

\paragraph{Research Replication Automation}

\paragraph{General Software Engineering LLM Agents}

\section{Conclusion}
\label{sec:conclusion}

\section*{Data Availability}
We share the anonymized version of~\approach~and~\benchmark{}: \todo.


\bibliographystyle{ACM-Reference-Format}
\bibliography{references}

\end{document}
\endinput
